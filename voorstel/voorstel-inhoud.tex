%---------- Inleiding ---------------------------------------------------------

% TODO: Is dit voorstel gebaseerd op een paper van Research Methods die je
% vorig jaar hebt ingediend? Heb je daarbij eventueel samengewerkt met een
% andere student?
% Zo ja, haal dan de tekst hieronder uit commentaar en pas aan.

%\paragraph{Opmerking}

% Dit voorstel is gebaseerd op het onderzoeksvoorstel dat werd geschreven in het
% kader van het vak Research Methods dat ik (vorig/dit) academiejaar heb
% uitgewerkt (met medesturent VOORNAAM NAAM als mede-auteur).
% 

\section{Inleiding}%
\label{sec:inleiding}

Veel web agencies gebruiken op dit moment verschillende losse tools voor het beheren van projecten, communicatie, en resources. Denk bijvoorbeeld aan tools zoals Trello voor takenbeheer, Slack voor communicatie, en Figma voor design. Hoewel deze tools nuttig zijn, ontbreekt het aan een centrale plek waar alles samenkomt. Dit leidt vaak tot inefficiënties, miscommunicatie en frustratie bij zowel medewerkers als klanten.
\\
\\
Dit onderzoeksproject heeft als doel een all-in-one platform te ontwikkelen dat specifiek gericht is op de behoeften van web agencies. Het platform biedt een geïntegreerde oplossing die alle noodzakelijke tools bevat. Naast standaardfunctionaliteiten zoals project- en takenbeheer, bevat het platform een unieke resource library waar medewerkers designs, Figma-bestanden en code snippets kunnen opslaan en live aanpassen. Deze mogelijkheid vereenvoudigt het samenwerken tussen ontwerpers en ontwikkelaars aanzienlijk.
\\
\\
Daarnaast is er een klantenzone, waar klanten toegang krijgen tot een overzicht van alle deliverables. Ze kunnen hier precies zien wat er wanneer gebeurt en voltooide deliverables bekijken of downloaden. Als een klant herhalingen nodig heeft van bepaalde onderdelen, kan dit eenvoudig worden aangevraagd via het platform. Dit verhoogt de transparantie en klanttevredenheid.
\\
\\
De centrale probleemstelling is: Hoe kan een gecentraliseerd platform de interne workflow van web agencies verbeteren en tegelijkertijd de transparantie naar klanten vergroten? De onderzoeksvraag die hieruit volgt is: Welke functionaliteiten moeten aanwezig zijn in een platform dat zowel interne teams als klanten ondersteunt in het projectproces?
\\
\\
Het einddoel is een werkend prototype dat getest kan worden door agencies en klanten. Het platform zal web agencies helpen hun werkprocessen te verbeteren, beter samen te werken en effectiever te communiceren met hun klanten.
%---------- Stand van zaken ---------------------------------------------------

\section{Literatuurstudie}%
\label{sec:literatuurstudie}

Er zijn verschillende tools beschikbaar voor projectbeheer en communicatie binnen web agencies, zoals Trello, Asana, en Slack, maar geen van deze biedt een geïntegreerde oplossing die specifiek gericht is op het beheren van design- en code-resources. De meeste platforms voor projectbeheer bieden geen mogelijkheid voor het live bewerken van code snippets of het delen van Figma-bestanden in een gestructureerde en collaboratieve manier. Het onderzoek van~\textcite{Reid2014} en~\textcite{Alexander2019} benadrukt dat de samenwerking tussen ontwerpers en ontwikkelaars vaak wordt verslechterd door het gebrek aan geïntegreerde tools.
\\
\\
Dit platform zal deze gat vullen door een systematische aanpak van zowel projectbeheer als resource management, waarbij de resource library een essentieel onderdeel is van de workflow. Er is weinig literatuur die ingaat op het combineren van projectmanagement, klantcommunicatie en klant deliverables met een geavanceerde resource library. Dit onderzoek zal deze nieuwe benadering van integratie verder onderzoeken.



% Voor literatuurverwijzingen zijn er twee belangrijke commando's:
% \autocite{KEY} => (Auteur, jaartal) Gebruik dit als de naam van de auteur
%   geen onderdeel is van de zin.
% \textcite{KEY} => Auteur (jaartal)  Gebruik dit als de auteursnaam wel een
%   functie heeft in de zin (bv. ``Uit onderzoek door Doll & Hill (1954) bleek
%   ...'')


%---------- Methodologie ------------------------------------------------------
\section{Methodologie}%
\label{sec:methodologie}

De ontwikkelingsmethode zal bestaan uit 3 fases, waarbij het platform in fasen wordt ontworpen, getest en verbeterd op basis van feedback van school.
\\
\\
In de eerste fase zal het ontwerp van de gebruikersinterface (UI) en gebruikerservaring (UX) plaatsvinden, waarbij de nadruk ligt op gebruiksvriendelijkheid en functionaliteit. In de tweede fase zal het platform worden ontwikkeld in Next.js voor de webversie en React Native voor de mobiele app. Supabase of neon zal worden gebruikt voor de opslag van gegevens, en Socket.io zal worden ingezet voor real-time samenwerking. In de derde fase zullen tests worden uitgevoerd om feedback te verzamelen en het platform te verbeteren.
\\
\\
De tools die gebruikt zullen worden, zijn onder andere Figma voor het ontwerp, Visual Studio Code voor de ontwikkeling, en GitHub voor versiebeheer. De tijdsplanning is als volgt:

\begin{itemize}
    \item Fase 1 (UI/UX ontwerp): 1 week
    \item Fase 2 (Platformontwikkeling): 9 weken
    \item Fase 3 (Testen en feedback): 2 weken
    
\end{itemize}

%---------- Verwachte resultaten ----------------------------------------------
\section{Verwacht resultaat, conclusie}%
\label{sec:verwachte_resultaten}

Het verwachte resultaat is een werkend platform dat takenbeheer, meetings, resourcebeheer, en klantinteractie combineert in één systeem. Het biedt een gestroomlijnde ervaring voor zowel medewerkers als klanten.Klanten zullen profiteren van een overzichtelijke klantenzone, terwijl medewerkers efficiënter kunnen samenwerken dankzij de resource library en real-time bewerkingsmogelijkheden.
\\
\\
Dit project zal agencies in staat stellen om hun processen te stroomlijnen, de communicatie te verbeteren, en de klanttevredenheid te verhogen. Het platform biedt niet alleen een oplossing voor bestaande problemen, maar ook een toekomstgerichte aanpak voor web agencies die willen groeien en innoveren.
