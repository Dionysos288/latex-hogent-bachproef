%==============================================================================
% Sjabloon onderzoeksvoorstel bachproef
%==============================================================================
% Gebaseerd op document class `hogent-article'
% zie <https://github.com/HoGentTIN/latex-hogent-article>

% Voor een voorstel in het Engels: voeg de documentclass-optie [english] toe.
% Let op: kan enkel na toestemming van de bachelorproefcoördinator!
\documentclass{hogent-article}

% Invoegen bibliografiebestand
\addbibresource{voorstel.bib}

% Informatie over de opleiding, het vak en soort opdracht
\studyprogramme{Professionele bachelor toegepaste informatica}
\course{Bachelorproef}
\assignmenttype{Onderzoeksvoorstel}
% Voor een voorstel in het Engels, haal de volgende 3 regels uit commentaar
% \studyprogramme{Bachelor of applied information technology}
% \course{Bachelor thesis}
% \assignmenttype{Research proposal}

\academicyear{2024-2025} % TODO: pas het academiejaar aan

% TODO: Werktitel
\title{Ontwikkeling van een all-in-one platform voor web agencies ter verbetering van communicatie, projectbeheer en resource management}

% TODO: Studentnaam en emailadres invullen
\author{Zeneli Dion}
\email{dion.zeneli@student.hogent.be}

% TODO: Medestudent
% Gaat het om een bachelorproef in samenwerking met een student in een andere
% opleiding? Geef dan de naam en emailadres hier
% \author{Yasmine Alaoui (naam opleiding)}
% \email{yasmine.alaoui@student.hogent.be}

% TODO: Geef de co-promotor op
% \supervisor[Co-promotor]{S. Beekman (Synalco, \href{mailto:sigrid.beekman@synalco.be}{sigrid.beekman@synalco.be})}

% Binnen welke specialisatierichting uit 3TI situeert dit onderzoek zich?
% Kies uit deze lijst:
%
% - Mobile \& Enterprise development
% - AI \& Data Engineering
% - Functional \& Business Analysis
% - System \& Network Administrator
% - Mainframe Expert
% - Als het onderzoek niet past binnen een van deze domeinen specifieer je deze
%   zelf
%
\specialisation{Mobile \& Enterprise development}
\keywords{Web Agencies, All-in-One Platform, Communicatie, Resource Library}

\begin{document}
    
    \begin{abstract}
      Dit onderzoeksvoorstel richt zich op het ontwikkelen van een all-in-one platform speciaal ontworpen voor web agencies. Het platform zal een centrale plaats bieden waar alle nodige tools aanwezig zijn voor het efficiënt beheren van projecten, taken, communicatie en resources. Het bevat functies zoals een uitgebreid takenbeheer, een systeem voor het plannen van vergaderingen, en een sectie voor het delen en beheren van design- en code-resources.
      \\
      \\
      Een belangrijk onderdeel is de resource library, waar medewerkers eenvoudig designs, Figma-bestanden en code snippets kunnen opslaan, bekijken en zelfs live aanpassen. Dit maakt het samenwerken tussen ontwerpers en ontwikkelaars efficiënter en intuïtiever. Daarnaast is er een speciaal gedeelte voor klanten waar ze toegang hebben tot alle geplande en voltooide deliverables. Klanten kunnen hier zien wat er op elk moment gebeurt, delivarables downloaden, en herhalingen van bepaalde delen aanvragen indien nodig.
      \\
      \\
      Het platform speelt in op de uitdaging waarmee veel agencies te maken hebben: het gebrek aan een gecentraliseerde en gestroomlijnde aanpak voor hun werkprocessen. De centrale probleemstelling is hoe agencies efficiënter kunnen werken en tegelijkertijd transparanter kunnen communiceren met hun klanten. De centrale onderzoeksvraag is: Hoe kan een geïntegreerd platform ontworpen worden dat de communicatie, projectbeheersing en het delen van resources binnen web agencies verbetert?
      \\
      \\
      De onderzoeksdoelstelling is het ontwikkelen van een werkend prototype van dit platform, dat zowel een web- als mobiele versie heeft. Verwachte resultaten zijn onder andere een functioneel platform dat de productiviteit verhoogt, de samenwerking binnen teams verbetert, en een betere relatie met klanten mogelijk maakt. Het project biedt een aanzienlijke meerwaarde door alle essentiële tools in één overzichtelijk systeem samen te brengen.
    \end{abstract}
    
    \tableofcontents
    
    % De hoofdtekst van het voorstel zit in een apart bestand, zodat het makkelijk
    % kan opgenomen worden in de bijlagen van de bachelorproef zelf.
    %---------- Inleiding ---------------------------------------------------------

% TODO: Is dit voorstel gebaseerd op een paper van Research Methods die je
% vorig jaar hebt ingediend? Heb je daarbij eventueel samengewerkt met een
% andere student?
% Zo ja, haal dan de tekst hieronder uit commentaar en pas aan.

%\paragraph{Opmerking}

% Dit voorstel is gebaseerd op het onderzoeksvoorstel dat werd geschreven in het
% kader van het vak Research Methods dat ik (vorig/dit) academiejaar heb
% uitgewerkt (met medesturent VOORNAAM NAAM als mede-auteur).
% 

\section{Inleiding}%
\label{sec:inleiding}

Veel web agencies gebruiken momenteel verschillende losse tools voor het beheren van projecten, communicatie, en resources. Denk bijvoorbeeld aan tools zoals Trello voor takenbeheer, Slack voor communicatie, en Figma voor design. Hoewel deze tools nuttig zijn, ontbreekt het vaak aan een centrale plek waar alles samenkomt. Dit leidt vaak tot inefficiënties, miscommunicatie en frustratie bij zowel medewerkers als klanten. Deze knelpunten komen voort uit het gebruik van verschillende systemen die niet altijd goed integreren, wat leidt tot extra werk, vertragingen en misverstanden.
\\
\\
Dit onderzoeksproject heeft als doel een all-in-one platform te ontwikkelen dat specifiek gericht is op de behoeften van web agencies. Het platform biedt een geïntegreerde oplossing die alle noodzakelijke tools bevat. Naast standaardfunctionaliteiten zoals project- en takenbeheer, bevat het platform een unieke resource library waar medewerkers designs, Figma-bestanden en code snippets kunnen opslaan en live aanpassen. Deze mogelijkheid vereenvoudigt het samenwerken tussen ontwerpers en ontwikkelaars aanzienlijk, wat de efficiëntie zou kunnen verbeteren.
\\
\\
Daarnaast is er een klantenzone, waar klanten toegang krijgen tot een overzicht van alle deliverables. Ze kunnen hier precies zien wat er wanneer gebeurt en voltooide deliverables bekijken of downloaden. Als een klant herhalingen nodig heeft van bepaalde onderdelen, kan dit eenvoudig worden aangevraagd via het platform. Dit verhoogt de transparantie en klanttevredenheid.
\\
\\
De centrale probleemstelling is: \textcolor{gray}{Hoe kan een gecentraliseerd platform de interne workflow van web agencies verbeteren en tegelijkertijd de transparantie naar klanten vergroten?}


\subsection{Deelvragen m.b.t. het probleemdomein}

De deelvragen met betrekking tot het probleemdomein zijn gericht op het identificeren van de specifieke bottlenecks in de processen en communicatie binnen web agencies. Deze deelvragen richten zich op het begrijpen van de interne werkstromen en de knelpunten die ontstaan door het gebruik van verschillende losse tools. Enkele mogelijke deelvragen zijn:

\begin{itemize}
    \item \textcolor{gray}{Wat zijn de specifieke bottlenecks die ontstaan door het gebruik van gescheiden tools voor projectbeheer, communicatie en resourcebeheer binnen web agencies?}
    \item \textcolor{gray}{Hoe beïnvloeden de gescheiden systemen de samenwerking tussen ontwerpers, ontwikkelaars en klanten? Welke communicatieproblemen ontstaan hierdoor?}
    \item \textcolor{gray}{In welke mate zorgen de integratieproblemen van tools voor vertragingen in het projectproces en onduidelijkheden in de klantcommunicatie?}
    \item \textcolor{gray}{Wat zijn de uitdagingen bij het beheren van design- en code-resources in een gefragmenteerde werkomgeving, en hoe beïnvloedt dit de productiviteit van medewerkers?}
    \item \textcolor{gray}{Hoe beïnvloedt het gebrek aan transparantie en overzicht voor klanten de klanttevredenheid en hun vertrouwen in het projectproces?}
\end{itemize}

\subsection{Deelvragen m.b.t. het oplossingsdomein}

De deelvragen met betrekking tot het oplossingsdomein richten zich op de kenmerken en vereisten van een platform dat deze bottlenecks en inefficiënties kan verhelpen. Deze vragen helpen om de belangrijkste functionaliteiten van het platform te bepalen die de workflow binnen web agencies kunnen verbeteren en tegelijkertijd de klanttevredenheid kunnen verhogen:

\begin{itemize}
    \item \textcolor{gray}{Welke functionaliteiten moeten er aanwezig zijn in een platform om een geïntegreerde oplossing te bieden voor projectbeheer, communicatie en resourcebeheer binnen web agencies?}
    \item \textcolor{gray}{Hoe kan een platform de samenwerking tussen ontwerpers en ontwikkelaars verbeteren door middel van een resource library voor live bewerking van Figma-bestanden en code snippets?}
    \item \textcolor{gray}{Welke elementen moeten er aanwezig zijn in een klantenzone om de transparantie en klanttevredenheid te verbeteren?}
    \item \textcolor{gray}{Hoe kan real-time samenwerking tussen medewerkers en klanten effectief worden geïmplementeerd in een platform?}
\end{itemize}

Het einddoel van dit project is een werkend prototype dat getest kan worden door agencies en klanten, met als doel de werkprocessen te verbeteren, de samenwerking te vergemakkelijken en de communicatie met klanten te optimaliseren.

%---------- Stand van zaken ---------------------------------------------------

\section{Literatuurstudie}%
\label{sec:literatuurstudie}
Er zijn verschillende tools beschikbaar voor projectbeheer en communicatie binnen web agencies, zoals Trello, Asana en Slack, maar geen van deze biedt een oplossing die alle functies samenbrengt. Veel van deze platforms bieden geen mogelijkheid om code snippets live te bewerken of Figma-bestanden op een georganiseerde manier te delen. Onderzoek van~\textcite{Reid2014} en~\textcite{Alexander2019} toont aan dat de samenwerking tussen ontwerpers en ontwikkelaars vaak moeilijker is door het gebruik van verschillende, niet-geïntegreerde tools.
\\
\\
Er is een duidelijke vraag naar een centrale plek waar verschillende functies voor projectbeheer, communicatie en resources samenkomen.~\textcite{Santoso2024} stelt dat zulke platforms niet alleen projectbeheer en taakorganisatie moeten bieden, maar ook samenwerking tussen verschillende teams, zoals ontwerpers en ontwikkelaars, moeten bevorderen. Het combineren van projectbeheer, klantcommunicatie en resourcebeheer in één systeem kan volgens~\textcite{Chris2024} de klanttevredenheid verbeteren en de werkdruk bij medewerkers verlichten.
\\
\\
Dit platform richt zich op het oplossen van deze problemen door een alles-in-één oplossing te bieden voor projectbeheer en resourcebeheer, met de resource library als een belangrijk onderdeel van de workflow. Er is weinig onderzoek dat de integratie van projectmanagement, klantcommunicatie en een geavanceerde resource library bespreekt. Dit onderzoek zal deze nieuwe benadering verder onderzoeken.


% Voor literatuurverwijzingen zijn er twee belangrijke commando's:
% \autocite{KEY} => (Auteur, jaartal) Gebruik dit als de naam van de auteur
%   geen onderdeel is van de zin.
% \textcite{KEY} => Auteur (jaartal)  Gebruik dit als de auteursnaam wel een
%   functie heeft in de zin (bv. ``Uit onderzoek door Doll & Hill (1954) bleek
%   ...'')


%---------- Methodologie ------------------------------------------------------
\section{Methodologie}%
\label{sec:methodologie}

De ontwikkelingsmethode zal bestaan uit 3 fases, waarbij het platform in fasen wordt ontworpen, getest en verbeterd op basis van feedback van school.
\\
\\
In de eerste fase zal het ontwerp van de gebruikersinterface(UI) en gebruikerservaring(UX) plaatsvinden, waarbij de nadruk ligt op gebruiksvriendelijkheid en functionaliteit. In de tweede fase zal het platform worden ontwikkeld in Next.js voor de webversie. Supabase of neon zal worden gebruikt voor de opslag van gegevens, en Socket.io zal worden ingezet voor real-time samenwerking. Tijdens de derde fase worden kwalitatieve interviews en surveys uitgevoerd met medewerkers van web agencies om te achterhalen in hoeverre het platform hun werkprocessen verbeterd. Ook worden A/B-tests uitgevoerd om de effectiviteit van specifieke functionaliteiten te evalueren.
\\
\\
De tools die gebruikt zullen worden, zijn onder andere Figma voor het ontwerp, Visual Studio Code voor de ontwikkeling, en GitHub voor versiebeheer. De tijdsplanning is als volgt:

\begin{itemize}
    \item Fase 1 (UI/UX ontwerp): 1 week
    \item Fase 2 (Platformontwikkeling): 9 weken
    \item Fase 3 (Testen en feedback): 2 weken
    
\end{itemize}

%---------- Verwachte resultaten ----------------------------------------------
\section{Verwacht resultaat, conclusie}%
\label{sec:verwachte_resultaten}

Het verwachte resultaat is een werkend platform dat takenbeheer, meetings, resourcebeheer, en klantinteractie combineert in één systeem. Het biedt een gestroomlijnde ervaring voor zowel medewerkers als klanten. Klanten zullen profiteren van een overzichtelijke klantenzone, terwijl medewerkers efficiënter kunnen samenwerken dankzij de resource library en real-time bewerkingsmogelijkheden.
\\
\\
Dit project zal agencies in staat stellen om hun processen te stroomlijnen, de communicatie te verbeteren, en de klanttevredenheid te verhogen. Het platform biedt niet alleen een oplossing voor bestaande problemen, maar ook een toekomstgerichte aanpak voor web agencies die willen groeien en innoveren.

    
    \printbibliography[heading=bibintoc]
    
\end{document}