%%=============================================================================
%% Inleiding
%%=============================================================================

\chapter{\IfLanguageName{dutch}{Inleiding}{Introduction}}
\label{ch:inleiding}

Web agencies worden geconfronteerd met steeds complexere uitdagingen in het beheren van hun projecten, resources en klantcommunicatie. Het \textcite{GitLab2023} DevSecOps rapport identificeert dat development teams heel erg worstelen met de integratie van verschillende tools en workflows in hun dagelijkse werk.

\section{Probleemstelling}
\label{sec:probleemstelling}

Moderne web agencies worstelen met drie kernproblemen in hun dagelijkse werking:

\begin{enumerate}
    \item \textbf{Verspreide workflows}: Het \textcite{StackOverflow2023} onderzoek toont aan dat ontwikkelaars een groot aantal verschillende tools moeten gebruiken voor hun dagelijkse taken. Deze niet samenhanging leidt tot aanzienlijk tijdverlies, miscommunicatie en verminderde productiviteit.
    
    \item \textbf{Inefficiënt resourcebeheer}: Het \textcite{GitHub2023} Octoverse rapport beschrijft hoe ontwikkelaars veel tijd verliezen aan het zoeken en hergebruiken van code en assets door het ontbreken van een samenhangend systeem.
    
    \item \textbf{Beperkte klanttransparantie}: Het \textcite{StateOfAgile2023} rapport toont aan dat organisaties grote uitdagingen ervaren met stakeholder communicatie en het effectief monitoren van projectvoortgang.
\end{enumerate}

\section{Onderzoeksvraag}
\label{sec:onderzoeksvraag}

De centrale onderzoeksvraag van deze bachelorproef is:

\begin{quote}
    Hoe kan een geïntegreerd platform worden ontwikkeld dat de workflows van web agencies optimaliseert door project management, resource beheer en klantcommunicatie te combineren in 1 samenhangend systeem?
\end{quote}

Deze hoofdvraag wordt ondersteund door de volgende deelvragen:

\begin{itemize}
    \item Welke functionaliteiten zijn essentieel voor effectief projectmanagement in web agencies?
    \item Hoe kan een resource library worden geïmplementeerd die code, design en documentatie effectief beheert?
    \item Welke features zijn nodig voor optimale klantcommunicatie en projecttransparantie?
    \item Hoe kunnen externe systemen (zoals Google Calendar) worden geïntegreerd voor maximale efficiency?
\end{itemize}

\section{Onderzoeksdoelstelling}
\label{sec:onderzoeksdoelstelling}

Het doel van dit onderzoek is het ontwikkelen van een samenhangend platform dat:

\begin{itemize}
    \item Project- en taakmanagement combineert met resource management
    \item Een uitgebreide resource library biedt voor code, design en documentatie
    \item Real-time collaboratie mogelijk maakt voor teamleden
    \item Een transparant klantportaal biedt met fase-tracking en revisie mogelijkheden
    \item Integreert met externe systemen zoals Google Calendar
\end{itemize}

\section{Opzet van deze bachelorproef}
\label{sec:opzet}

Deze bachelorproef is als volgt gestructureerd:

\begin{description}
    \item[Stand van zaken] Analyseert de huidige tools en praktijken in web agencies, identificeert tekortkomingen en onderzoekt moderne oplossingen.
    
    \item[Methodologie] Beschrijft de ontwikkelingsaanpak, gebruikte technologieën en evaluatiemethoden.
    
    \item[Implementatie] Presenteert de technische realisatie van het platform, inclusief architectuur en kernfunctionaliteiten.
    
    \item[Evaluatie] Bespreekt de resultaten van gebruikerstests en vergelijkt het platform met bestaande oplossingen.
    
    \item[Conclusie] Vat de belangrijkste bevindingen samen en doet aanbevelingen voor toekomstig onderzoek.
\end{description}

Dit onderzoek draagt bij aan de ontwikkeling van efficiëntere werkprocessen binnen web agencies en biedt een basis voor verdere innovatie in samenhangende ontwikkelplatforms.