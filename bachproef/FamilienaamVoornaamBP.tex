%===============================================================================
% LaTeX sjabloon voor de bachelorproef toegepaste informatica aan HOGENT
% Meer info op https://github.com/HoGentTIN/latex-hogent-report
%===============================================================================

\documentclass[dutch,dit,thesis]{hogentreport}

% TODO:
% - If necessary, replace the option `dit`' with your own department!
%   Valid entries are dbo, dbt, dgz, dit, dlo, dog, dsa, soa
% - If you write your thesis in English (remark: only possible after getting
%   explicit approval!), remove the option "dutch," or replace with "english".

\usepackage{lipsum} % For blind text, can be removed after adding actual content

%% Pictures to include in the text can be put in the graphics/ folder
\graphicspath{{../graphics/}}

%% For source code highlighting, requires pygments to be installed
%% Compile with the -shell-escape flag!
%% \usepackage[chapter]{minted}
%% If you compile with the make_thesis.{bat,sh} script, use the following
%% import instead:
\usepackage[chapter,outputdir=../output]{minted}
\usemintedstyle{solarized-light}

%% Formatting for minted environments.
\setminted{%
    autogobble,
    frame=lines,
    breaklines,
    linenos,
    tabsize=4
}

%% Ensure the list of listings is in the table of contents
\renewcommand\listoflistingscaption{%
    \IfLanguageName{dutch}{Lijst van codefragmenten}{List of listings}
}
\renewcommand\listingscaption{%
    \IfLanguageName{dutch}{Codefragment}{Listing}
}
\renewcommand*\listoflistings{%
    \cleardoublepage\phantomsection\addcontentsline{toc}{chapter}{\listoflistingscaption}%
    \listof{listing}{\listoflistingscaption}%
}

% Other packages not already included can be imported here

%%---------- Document metadata -------------------------------------------------
% TODO: Replace this with your own information
\author{Zeneli Dion}
\supervisor{Martine Van Audenrode}
\cosupervisor{Jo Van Snick}
\title{Ontwikkeling van een all-in-one platform voor web agencies ter verbetering van communicatie, projectbeheer en resource management}
\academicyear{\advance\year by -1 \the\year--\advance\year by 1 \the\year}
\examperiod{1}
\degreesought{\IfLanguageName{dutch}{Professionele bachelor in de toegepaste informatica}{Bachelor of applied computer science}}
\partialthesis{false} %% To display 'in partial fulfilment'
%\institution{Internshipcompany BVBA.}

%% Add global exceptions to the hyphenation here
\hyphenation{back-slash}

%% The bibliography (style and settings are  found in hogentthesis.cls)
\addbibresource{bachproef.bib}            %% Bibliography file
\addbibresource{../voorstel/voorstel.bib} %% Bibliography research proposal
\defbibheading{bibempty}{}

%% Prevent empty pages for right-handed chapter starts in twoside mode
\renewcommand{\cleardoublepage}{\clearpage}

\renewcommand{\arraystretch}{1.2}

%% Content starts here.
\begin{document}

%---------- Front matter -------------------------------------------------------

\frontmatter

\hypersetup{pageanchor=false} %% Disable page numbering references
%% Render a Dutch outer title page if the main language is English
\IfLanguageName{english}{%
    %% If necessary, information can be changed here
    \degreesought{Professionele Bachelor toegepaste informatica}%
    \begin{otherlanguage}{dutch}%
       \maketitle%
    \end{otherlanguage}%
}{}

%% Generates title page content
\maketitle
\hypersetup{pageanchor=true}

%%=============================================================================
%% Voorwoord
%%=============================================================================

\chapter*{\IfLanguageName{dutch}{Woord vooraf}{Preface}}%
\label{ch:voorwoord}


% Tijdens mijn stages en projecten als student toegepaste informatica heb ik gezien hoe moeilijk het kan zijn om met verschillende tools te werken in web agencies. Ontwikkelaars moeten constant wisselen tussen verschillende programma's voor projectbeheer, code schrijven en communicatie met klanten. Dit kost veel tijd en zorgt vaak voor verwarring.

% Dit was voor mij de reden om een platform te maken dat alles samenbrengt in één systeem. Ik wilde iets maken dat het werk makkelijker maakt voor zowel de ontwikkelaars als de klanten.

% Deze bachelorproef was niet mogelijk geweest zonder de hulp van verschillende mensen. Eerst wil ik mijn promotor, Mvr. Martine Van Audenrode, bedanken voor haar goede advies en feedback. Ook wil ik mijn co-promotor, Mr. Jo Van Snick, bedanken voor zijn hulp bij het maken van een gebruiksvriendelijk systeem.

% Ook ben ik dankbaar voor de web agencies die hebben meegeholpen met het testen van het platform. Hun feedback heeft echt geholpen om het systeem beter te maken.

% Als laatste wil ik mijn familie en vrienden bedanken voor hun steun tijdens dit project. Zonder hun aanmoediging was het een stuk lastiger geweest om dit project af te maken.

% \vspace{\baselineskip}

% Dion Zeneli\\\
% Gent, \today
%%=============================================================================
%% Samenvatting
%%=============================================================================

% TODO: De "abstract" of samenvatting is een kernachtige (~ 1 blz. voor een
% thesis) synthese van het document.
%
% Een goede abstract biedt een kernachtig antwoord op volgende vragen:
%
% 1. Waarover gaat de bachelorproef?
% 2. Waarom heb je er over geschreven?
% 3. Hoe heb je het onderzoek uitgevoerd?
% 4. Wat waren de resultaten? Wat blijkt uit je onderzoek?
% 5. Wat betekenen je resultaten? Wat is de relevantie voor het werkveld?
%
% Daarom bestaat een abstract uit volgende componenten:
%
% - inleiding + kaderen thema
% - probleemstelling
% - (centrale) onderzoeksvraag
% - onderzoeksdoelstelling
% - methodologie
% - resultaten (beperk tot de belangrijkste, relevant voor de onderzoeksvraag)
% - conclusies, aanbevelingen, beperkingen
%
% LET OP! Een samenvatting is GEEN voorwoord!

%%---------- Nederlandse samenvatting -----------------------------------------
%
% TODO: Als je je bachelorproef in het Engels schrijft, moet je eerst een
% Nederlandse samenvatting invoegen. Haal daarvoor onderstaande code uit
% commentaar.
% Wie zijn bachelorproef in het Nederlands schrijft, kan dit negeren, de inhoud
% wordt niet in het document ingevoegd.

\IfLanguageName{english}{%
\selectlanguage{dutch}
\chapter*{Samenvatting}
\lipsum[1-4]
\selectlanguage{english}
}{}

%%---------- Samenvatting -----------------------------------------------------
% De samenvatting in de hoofdtaal van het document

\chapter*{\IfLanguageName{dutch}{Samenvatting}{Abstract}}

Deze bachelorproef richt zich op het ontwikkelen van een samenhangend platform voor web agencies, met als doel het optimaliseren van projectmanagement, resourcebeheer en klantcommunicatie. Web agencies werken traditioneel met een versnipperd landschap van tools voor projectbeheer, code, design, documentatie en communicatie, wat leidt tot tijdsverlies, inefficiëntie en een gebrek aan overzicht. De centrale onderzoeksvraag luidde: \emph{Hoe kan een geïntegreerd platform worden ontwikkeld dat de workflows van web agencies optimaliseert door project management, resource beheer en klantcommunicatie te combineren in één samenhangend systeem?}

Om deze vraag te beantwoorden werd gestart met een grondige literatuurstudie naar de huidige stand van zaken in de sector. Uit deze studie bleek dat bestaande oplossingen zoals Jira, Trello, Figma, Notion en ClickUp elk slechts een deel van de workflow ondersteunen en vaak onvoldoende integratie bieden. De belangrijkste pijnpunten zijn gefragmenteerde workflows, inefficiënt resourcebeheer, beperkte klanttransparantie en het ontbreken van een centrale plek voor alle projectinformatie.

Op basis van deze inzichten werd een proof-of-concept platform ontwikkeld met de volgende kernfunctionaliteiten:
\begin{itemize}
    \item \textbf{Project- en taakmanagement:} Een intuïtief dashboard met drag-and-drop, inline editing, bulkacties en templates voor veelvoorkomende workflows.
    \item \textbf{Resource library:} Centrale opslag en beheer van code snippets (met Monaco Editor), design assets (met Figma-integratie en Cloudflare Images) en documenten (met Slate rich text editor en versiebeheer).
    \item \textbf{Team- en klantenzones:} Afgeschermde omgevingen voor teams en klanten, met strikte toegangscontrole en privacy.
    \item \textbf{Realtime notificaties:} Directe updates bij wijzigingen in projecten, taken en resources via Supabase Realtime.
    \item \textbf{Naadloze integraties:} Koppelingen met Google Calendar, Figma en Cloudflare Images voor een vloeiende workflow.
\end{itemize}

De methodologie bestond uit een herhalende ontwikkelproces, waarbij telkens kleine deelmodules werden uitgewerkt, getest en bijgestuurd op basis van feedback van testgebruikers, collega-studenten en de co-promotor. Zowel functionele als usability tests werden uitgevoerd, met aandacht voor gebruiksgemak, performance, security en schaalbaarheid. De testomgeving simuleerde een realistische agency omgeving met verschillende rollen en projecten.

Uit de analyse van de testresultaten blijkt dat het platform een duidelijke meerwaarde biedt ten opzichte van bestaande oplossingen. Gebruikers gaven aan dat zij minder tijd kwijt waren aan het wisselen tussen tools, sneller konden werken dankzij local-first synchronisatie en direct feedback kregen bij bewerkingen. De centrale resource library en de mogelijkheid om samen te werken aan code, design en documentatie werden als grote voordelen ervaren. Ook de klantenzone en realtime notificaties droegen bij aan een betere samenwerking en transparantie.

Hoewel het platform in hoge mate voldoet aan de gestelde requirements, zijn er ook enkele beperkingen en aandachtspunten. Zo zijn er nog geen geautomatiseerde end-to-end tests of security audits uitgevoerd, en is de evaluatie vooral gebaseerd op kwalitatieve feedback van een beperkte groep gebruikers. Gebruikers gaven aan dat zij in de toekomst graag meer integraties, mobiele ondersteuning, geavanceerdere zoek- en filtermogelijkheden en AI-ondersteunde features zouden zien.

De bachelorproef levert een werkend proof-of-concept en een blauwdruk voor hoe moderne web agencies hun workflows kunnen stroomlijnen. De combinatie van projectmanagement, resourcebeheer en klantcommunicatie in één platform resulteert in efficiëntere processen, betere samenwerking en meer tevreden klanten. Verdere optimalisatie, uitbreiding van functionaliteit en grootschalige praktijkproeven zullen de impact op het werkveld nog verder vergroten.


%---------- Inhoud, lijst figuren, ... -----------------------------------------

\tableofcontents

% In a list of figures, the complete caption will be included. To prevent this,
% ALWAYS add a short description in the caption!
%
%  \caption[short description]{elaborate description}
%
% If you do, only the short description will be used in the list of figures

\listoffigures

% If you included tables and/or source code listings, uncomment the appropriate
% lines.
\listoftables

\listoflistings

% Als je een lijst van afkortingen of termen wil toevoegen, dan hoort die
% hier thuis. Gebruik bijvoorbeeld de ``glossaries'' package.
% https://www.overleaf.com/learn/latex/Glossaries

%---------- Kern ---------------------------------------------------------------

\mainmatter{}

% De eerste hoofdstukken van een bachelorproef zijn meestal een inleiding op
% het onderwerp, literatuurstudie en verantwoording methodologie.
% Aarzel niet om een meer beschrijvende titel aan deze hoofdstukken te geven of
% om bijvoorbeeld de inleiding en/of stand van zaken over meerdere hoofdstukken
% te verspreiden!

%%=============================================================================
%% Inleiding
%%=============================================================================

\chapter{\IfLanguageName{dutch}{Inleiding}{Introduction}}
\label{ch:inleiding}

Web agencies worden geconfronteerd met steeds complexere uitdagingen in het beheren van hun projecten, resources en klantcommunicatie. Het \textcite{GitLab2023} DevSecOps rapport identificeert dat development teams heel erg worstelen met de integratie van verschillende tools en workflows in hun dagelijkse werk.

\section{Probleemstelling}
\label{sec:probleemstelling}

Moderne web agencies worstelen met drie kernproblemen in hun dagelijkse werking:

\begin{enumerate}
    \item \textbf{Verspreide workflows}: Het \textcite{StackOverflow2023} onderzoek toont aan dat ontwikkelaars een groot aantal verschillende tools moeten gebruiken voor hun dagelijkse taken. Deze niet samenhanging leidt tot aanzienlijk tijdverlies, miscommunicatie en verminderde productiviteit.
    
    \item \textbf{Inefficiënt resourcebeheer}: Het \textcite{GitHub2023} Octoverse rapport beschrijft hoe ontwikkelaars veel tijd verliezen aan het zoeken en hergebruiken van code en assets door het ontbreken van een samenhangend systeem.
    
    \item \textbf{Beperkte klanttransparantie}: Het \textcite{StateOfAgile2023} rapport toont aan dat organisaties grote uitdagingen ervaren met stakeholder communicatie en het effectief monitoren van projectvoortgang.
\end{enumerate}

\section{Onderzoeksvraag}
\label{sec:onderzoeksvraag}

De centrale onderzoeksvraag van deze bachelorproef is:

\begin{quote}
    Hoe kan een geïntegreerd platform worden ontwikkeld dat de workflows van web agencies optimaliseert door project management, resource beheer en klantcommunicatie te combineren in 1 samenhangend systeem?
\end{quote}

Deze hoofdvraag wordt ondersteund door de volgende deelvragen:

\begin{itemize}
    \item Welke functionaliteiten zijn essentieel voor effectief projectmanagement in web agencies?
    \item Hoe kan een resource library worden geïmplementeerd die code, design en documentatie effectief beheert?
    \item Welke features zijn nodig voor optimale klantcommunicatie en projecttransparantie?
    \item Hoe kunnen externe systemen (zoals Google Calendar) worden geïntegreerd voor maximale efficiency?
\end{itemize}

\section{Onderzoeksdoelstelling}
\label{sec:onderzoeksdoelstelling}

Het doel van dit onderzoek is het ontwikkelen van een samenhangend platform dat:

\begin{itemize}
    \item Project- en taakmanagement combineert met resource management
    \item Een uitgebreide resource library biedt voor code, design en documentatie
    \item Real-time collaboratie mogelijk maakt voor teamleden
    \item Een transparant klantportaal biedt met fase-tracking en revisie mogelijkheden
    \item Integreert met externe systemen zoals Google Calendar
\end{itemize}

\section{Opzet van deze bachelorproef}
\label{sec:opzet}

Deze bachelorproef is als volgt gestructureerd:

\begin{description}
    \item[Stand van zaken] Analyseert de huidige tools en praktijken in web agencies, identificeert tekortkomingen en onderzoekt moderne oplossingen.
    
    \item[Methodologie] Beschrijft de ontwikkelingsaanpak, gebruikte technologieën en evaluatiemethoden.
    
    \item[Implementatie] Presenteert de technische realisatie van het platform, inclusief architectuur en kernfunctionaliteiten.
    
    \item[Evaluatie] Bespreekt de resultaten van gebruikerstests en vergelijkt het platform met bestaande oplossingen.
    
    \item[Conclusie] Vat de belangrijkste bevindingen samen en doet aanbevelingen voor toekomstig onderzoek.
\end{description}

Dit onderzoek draagt bij aan de ontwikkeling van efficiëntere werkprocessen binnen web agencies en biedt een basis voor verdere innovatie in samenhangende ontwikkelplatforms.
\chapter{\IfLanguageName{dutch}{Stand van zaken}{State of the art}}
\label{ch:stand-van-zaken}

Dit hoofdstuk presenteert een grondige analyse van de huidige staat van project- en resourcemanagement binnen web agencies. Het onderzoek richt zich op het vinden van de grootste problemen, evaluatie van bestaande oplossingen en analyse van toekomstige ontwikkelingsmogelijkheden. Deze literatuurstudie vormt de basis voor de ontwikkeling van een samenwerkend platform.

\section{Huidige situatie in web agencies}
\label{sec:huidige-situatie}

De hedendaagse web development sector wordt gekenmerkt door toenemende moeilijkheden in projectmanagement en resource allocatie. Het \textcite{GitLab2023} DevSecOps rapport identificeert verschillende kritieke uitdagingen waarmee development teams dagelijks worden geconfronteerd:

\begin{itemize}
    \item \textbf{Tool fragmentatie}: Teams moeten schakelen tussen veel verschillende systemen
    \item \textbf{Workflow onderbrekingen}: Constante context-switching tussen tools
    \item \textbf{Kennisverspreiding}: Informatie verspreid over meerdere platforms
\end{itemize}

\subsection{Project en task management}
\label{subsec:project-management}

Het \textcite{Atlassian2023} rapport over Agile practices toont verschillende uitdagingen in modern projectmanagement:

\begin{itemize}
    \item \textbf{Project Planning}
    \begin{itemize}
        \item Moeilijkheden bij het opstellen van realistische timelines
        \item Uitdagingen in het definiëren van project scopes
        \item Complexiteit van budget allocatie
        \item Gebrek aan inzicht in resource beschikbaarheid
    \end{itemize}
    
    \item \textbf{Task Management}
    \begin{itemize}
        \item Uitdagingen in het prioriteren van taken
        \item Problemen met het tracken van afhankelijkheden
        \item Moeilijkheden bij het balanceren van workload
        \item Gebrek aan duidelijke task ownership
    \end{itemize}
    
    \item \textbf{Team Resource Allocatie}
    \begin{itemize}
        \item Complexiteit van skill-matching met taken
        \item Uitdagingen in capacity planning
        \item Moeilijkheden bij het managen van parallelle projecten
        \item Gebrek aan flexibiliteit bij onverwachte wijzigingen
    \end{itemize}
    
    \item \textbf{Project Monitoring}
    \begin{itemize}
        \item Beperkt inzicht in real-time projectstatus
        \item Moeilijkheden bij het tracken van milestones
        \item Uitdagingen in het meten van project gezondheid
        \item Gebrek aan vroege waarschuwingssystemen
    \end{itemize}
\end{itemize}

\subsection{Resource Library Management}
\label{subsec:resource-library}

Het \textcite{GitHub2023} Octoverse rapport toont de cruciale rol van een goed georganiseerde resource library:

\begin{itemize}
    \item \textbf{Code Management}
    \begin{itemize}
        \item Uitdagingen in het organiseren van herbruikbare componenten
        \item Moeilijkheden bij het vinden van relevante code snippets
        \item Gebrek aan context bij bestaande oplossingen
        \item Versie beheer van gedeelde componenten
        \item Documentatie van code gebruik en implementatie
    \end{itemize}
    
    \item \textbf{Design Asset Management}
    \begin{itemize}
        \item Complexiteit van design system versioning
        \item Uitdagingen in het beheren van brand assets
        \item Moeilijkheden bij het delen van UI componenten
        \item Gebrek aan consistentie in asset gebruik
        \item Inefficiënt zoeken naar specifieke assets
    \end{itemize}
    
    \item \textbf{Document Management}
    \begin{itemize}
        \item Uitdagingen in template standards
        \item Moeilijkheden bij het tracken van document versies
        \item Gebrek aan gestructureerde metadata
        \item Inefficiënte zoek- en filtermogelijkheden
        \item Beperkte mogelijkheden voor collaborative editing
    \end{itemize}
\end{itemize}

\subsection{Integratie Uitdagingen}
\label{subsec:integratie}

Het \textcite{StackOverflow2023} onderzoek identificeert specifieke moeilijkheden bij het samenbrengen van verschillende systemen:

\begin{itemize}
    \item \textbf{Data Synchronisatie}
    \begin{itemize}
        \item Inconsistenties tussen verschillende platforms
        \item Uitdagingen in real-time updates
        \item Complexiteit van data mapping
        \item Moeilijkheden bij het behouden van data integriteit
    \end{itemize}
    
    \item \textbf{Workflow Integratie}
    \begin{itemize}
        \item Gebrek aan naadloze overgangen tussen tools
        \item Uitdagingen in het automatiseren van processen
        \item Moeilijkheden bij het tracken van cross-tool activiteiten
        \item Beperkte mogelijkheden voor custom workflows
    \end{itemize}
    
    \item \textbf{User Experience}
    \begin{itemize}
        \item Inconsistente interfaces tussen verschillende tools
        \item Complexe navigatie tussen systemen
        \item Verwarrende notificatie systemen
        \item Gebrek aan unified search functionaliteit
    \end{itemize}
\end{itemize}

\section{Bestaande Oplossingen}
\label{sec:bestaande-oplossingen}

Het \textcite{GitLab2023} rapport analyseert de huidige markt van project en resource management tools:

\begin{itemize}
    \item \textbf{Project Management Platforms}
    \begin{itemize}
        \item Jira: Krachtig maar complex voor kleine teams
        \item Trello: Gebruiksvriendelijk maar beperkt in functionaliteit
        \item Asana: Goed voor task management, zwak in resource beheer
        \item Monday: Visueel sterk maar beperkt in technische integraties
        \item ClickUp: Veelzijdig maar overweldigend in opties
    \end{itemize}
    
    \item \textbf{Resource Management Tools}
    \begin{itemize}
        \item Figma: Sterk in design maar beperkt in andere resources
        \item GitHub: Uitstekend voor code, zwak in andere assets
        \item Notion: Goed voor documenten, beperkt in technische assets
        \item Confluence: Krachtig voor documentatie, complex in gebruik
        \item SharePoint: Uitgebreid maar niet ontwikkelaar-vriendelijk
    \end{itemize}
\end{itemize}

\section{Toekomstige Ontwikkelingen}
\label{sec:toekomst}

Het \textcite{StateOfAgile2023} rapport toont belangrijke trends voor project en resource management:

\begin{itemize}
    \item \textbf{Project Management Innovaties}
    \begin{itemize}
        \item AI-ondersteunde planning en estimatie
        \item Predictieve analytics voor projectrisico's
        \item Geautomatiseerde resource allocatie
        \item Smart workload balancing
        \item Contextbewuste task prioritering
    \end{itemize}
    
    \item \textbf{Resource Library Evolutie}
    \begin{itemize}
        \item AI-powered resource tagging en categorisatie
        \item Automatische versie controle en conflict resolutie
        \item Smart search en aanbevelingen
        \item Real-time collaborative editing
        \item Geautomatiseerde metadata generatie
    \end{itemize}
    
    \item \textbf{Integratie Verbeteringen}
    \begin{itemize}
        \item Unified platforms voor alle projectresources
        \item Naadloze cross-tool workflows
        \item Geautomatiseerde data synchronisatie
        \item Contextbewuste navigatie tussen tools
        \item Intelligente notificatie systemen
    \end{itemize}
\end{itemize}

\section{Conclusie Literatuurstudie}
\label{sec:conclusie}

Deze literatuurstudie toont aan dat er een duidelijke nood is aan een samenwerkend platform voor web agencies. De huidige niet samenwerkende aanpak leidt tot significante vertragingen in projectmanagement, resource beheer en klantcommunicatie. Een succesvol platform moet de volgende aspecten combineren:

\begin{itemize}
    \item \textbf{Geïntegreerd Project Management}
    \begin{itemize}
        \item Intuïtieve planning en tracking
        \item Flexibele resource allocatie
        \item Real-time voortgangsmonitoring
        \item Geautomatiseerde workflows
    \end{itemize}
    
    \item \textbf{Centrale Resource Library}
    \begin{itemize}
        \item Unified storage voor alle assets
        \item Krachtige zoek- en filterfuncties
        \item Gestructureerd versie beheer
        \item Intelligente categorisatie
    \end{itemize}
    
    \item \textbf{Naadloze Integratie}
    \begin{itemize}
        \item Consistente gebruikerservaring
        \item Geautomatiseerde synchronisatie
        \item Contextbewuste navigatie
        \item Unified search across resources
    \end{itemize}
\end{itemize}

Deze inzichten vormen de basis voor de ontwikkeling van het platform dat in de volgende hoofdstukken wordt beschreven. De focus ligt daarbij op het creëren van een samenwerkende oplossing die de gevonden problemen adresseert en aan de moderne eisen van web development teams voldoet.


%%=============================================================================
%% Methodologie
%%=============================================================================

\chapter{\IfLanguageName{dutch}{Methodologie}{Methodology}}
\label{ch:methodologie}

Dit hoofdstuk beschrijft de methodologische aanpak die is gebruikt voor het onderzoek en de ontwikkeling van het geïntegreerde platform. Het onderzoek is uitgevoerd in verschillende fasen, elk met specifieke doelstellingen en methoden.

\section{Onderzoeksfasen}
\label{sec:onderzoeksfasen}

Het onderzoek is opgedeeld in vijf hoofdfasen:

\begin{enumerate}
    \item Literatuuronderzoek en requirements analyse
    \item Design en architectuur planning
    \item Ontwikkeling en implementatie
    \item Testing en evaluatie
    \item Analyse en conclusievorming
\end{enumerate}

\section{Fase 1: Literatuuronderzoek en Requirements}
\label{sec:fase1}

De eerste fase bestond uit een uitgebreid literatuuronderzoek naar bestaande oplossingen en best practices. Hierbij zijn verschillende onderzoeksmethoden toegepast:

\begin{itemize}
    \item Systematische review van academische literatuur via IEEE Xplore en ACM Digital Library
    \item Analyse van bestaande platforms en hun beperkingen
    \item Interviews met potentiële eindgebruikers voor requirements gathering
\end{itemize}

Deze fase resulteerde in een gedetailleerd requirements document en een duidelijk beeld van de huidige stand van zaken in het vakgebied.

\section{Fase 2: Design en Architectuur}
\label{sec:fase2}

De tweede fase focuste op het ontwerp van de technische architectuur en gebruikersinterface:

\begin{itemize}
    \item Ontwikkeling van technische architectuur gebaseerd op Next.js
    \item Design van database schema's met Prisma
    \item Opstellen van UI/UX wireframes en prototypes
    \item Planning van integraties met externe systemen
\end{itemize}

Hierbij is gebruik gemaakt van moderne design methodologieën zoals User-Centered Design en iteratieve feedback sessies met stakeholders.

\section{Fase 3: Ontwikkeling}
\label{sec:fase3}

De ontwikkelingsfase volgde een agile aanpak met twee-wekelijkse sprints:

\begin{itemize}
    \item Implementatie van core functionaliteiten:
    \begin{itemize}
        \item Project en task management systeem
        \item Resource library met code editor
        \item Klantportaal met fase-tracking
        \item Calendar integratie
    \end{itemize}
    \item Continuous integration en deployment pipeline
    \item Code reviews en technische documentatie
\end{itemize}

\section{Fase 4: Testing en Evaluatie}
\label{sec:fase4}

De evaluatiefase bestond uit verschillende testmethodologieën:

\begin{itemize}
    \item Functionele tests van alle componenten
    \item Usability testing met eindgebruikers
    \item Performance testing van kritieke functionaliteiten
    \item Security audits van het platform
\end{itemize}

Feedback werd verzameld via:
\begin{itemize}
    \item Gestructureerde gebruikersinterviews
    \item System Usability Scale (SUS) vragenlijsten
    \item Performance metrics analyse
\end{itemize}

\section{Fase 5: Analyse en Conclusie}
\label{sec:fase5}

De laatste fase focuste op het analyseren van de verzamelde data en het trekken van conclusies:

\begin{itemize}
    \item Analyse van gebruikersfeedback en testresultaten
    \item Vergelijking met initiële requirements
    \item Identificatie van verbeterpunten en aanbevelingen
    \item Documentatie van bevindingen en conclusies
\end{itemize}

Deze methodologische aanpak heeft geleid tot een systematische ontwikkeling en evaluatie van het platform, waarbij elke fase heeft bijgedragen aan het beantwoorden van de onderzoeksvragen. De gedetailleerde resultaten van elke fase worden in de volgende hoofdstukken besproken.



\chapter{Casus: Vergelijkende Analyse met Bestaande Applicaties}
\label{ch:casus-vergelijking}

In dit hoofdstuk wordt een concrete casus uitgewerkt waarin het ontwikkelde platform wordt vergeleken met enkele toonaangevende bestaande applicaties die veel gebruikt worden binnen web agencies, zoals Jira, Trello, Figma, Notion en ClickUp. Het doel is om de praktische meerwaarde, de efficiëntie en de gebruikerservaring van het eigen platform te toetsen aan de hand van een realistisch projectscenario.

\section{Beschrijving van de casus}
\label{sec:casus-beschrijving}

Stel: een web agency krijgt de opdracht om voor een klant een nieuwe bedrijfswebsite te ontwerpen en te ontwikkelen. Het project omvat verschillende fases: intake en analyse, design, development, testing en oplevering. Het team bestaat uit een projectmanager, twee developers, een designer en een klant die de voortgang opvolgt en feedback geeft.

\section{Werken met traditionele tools}
\label{sec:casus-traditioneel}

In een klassieke workflow worden verschillende tools ingezet:
\begin{itemize}
    \item \textbf{Jira} voor het plannen en opvolgen van taken en sprints.
    \item \textbf{Figma} voor het ontwerpen van wireframes en UI-componenten.
    \item \textbf{GitHub} voor het beheren van code en versiecontrole.
    \item \textbf{Notion} voor het documenteren van projectafspraken, requirements en notulen.
    \item \textbf{E-mail/Slack} voor communicatie met de klant en het delen van updates.
\end{itemize}

Deze aanpak leidt tot de volgende uitdagingen:
\begin{itemize}
    \item Teamleden moeten voortdurend schakelen tussen verschillende applicaties, wat tijd kost en het risico op fouten vergroot.
    \item Informatie raakt verspreid: taken in Jira, designs in Figma, documentatie in Notion, code in GitHub, communicatie in Slack of e-mail.
    \item Klanten hebben geen centraal overzicht van de voortgang en moeten voor feedback verschillende kanalen gebruiken.
    \item Het zoeken naar de juiste informatie of bestanden kost veel tijd, vooral bij grotere projecten.
    \item Het bijhouden van de status van taken, assets en feedback is omslachtig en niet altijd up-to-date.
\end{itemize}

\section{Werken met het ontwikkelde platform}
\label{sec:casus-eigen-platform}

In het eigen platform worden alle bovenstaande aspecten samengebracht in één geïntegreerde omgeving:
\begin{itemize}
    \item \textbf{Project- en taakbeheer:} Alle taken, fases en deadlines zijn zichtbaar in een centraal dashboard met drag-and-drop, inline editing en bulkacties.
    \item \textbf{Resource library:} Code snippets, design assets (inclusief Figma-integratie en Cloudflare Images) en documenten zijn centraal opgeslagen, doorzoekbaar en versieerbaar.
    \item \textbf{Klantenzone:} De klant heeft een eigen portaal waar hij de voortgang kan volgen, feedback kan geven en documenten kan inzien.
    \item \textbf{Realtime notificaties:} Alle teamleden en de klant ontvangen direct updates bij relevante wijzigingen.
    \item \textbf{Integraties:} Google Calendar, Figma en Cloudflare Images zijn naadloos geïntegreerd, waardoor planning en resourcebeheer automatisch verlopen.
    \item \textbf{Samenwerking:} Teamleden kunnen in real-time samenwerken aan documenten, taken en assets zonder de context van het project te verliezen.
\end{itemize}

\section{Vergelijkende analyse: voordelen en beperkingen}
\label{sec:casus-analyse}

\subsection{Voordelen van het eigen platform}
\begin{itemize}
    \item \textbf{Efficiëntie:} Minder tijdverlies door context-switching; alle informatie is centraal beschikbaar.
    \item \textbf{Overzicht:} Eén dashboard voor alle projectinformatie, taken, assets en communicatie.
    \item \textbf{Transparantie:} Klanten zijn continu op de hoogte van de voortgang en kunnen eenvoudig feedback geven.
    \item \textbf{Samenwerking:} Real-time updates en centrale opslag bevorderen samenwerking en kennisdeling.
    \item \textbf{Gebruiksgemak:} Intuïtieve interface, snelle interacties dankzij local-first synchronisatie en bulkacties.
    \item \textbf{Beheer:} Eenvoudig rechtenbeheer en duidelijke scheiding tussen teams en klantenzones.
\end{itemize}

\subsection{Beperkingen en aandachtspunten}
\begin{itemize}
    \item \textbf{Integraties:} Hoewel de belangrijkste integraties aanwezig zijn, missen sommige agencies mogelijk koppelingen met andere tools zoals Slack of Jira.
    \item \textbf{Adoptie:} Teams die gewend zijn aan bestaande tools moeten wennen aan een nieuwe workflow.
    \item \textbf{Schaalbaarheid:} Het platform is nog niet grootschalig getest met honderden gebruikers of zeer grote projecten.
    \item \textbf{Mobiele ondersteuning:} Een mobiele app of geoptimaliseerde mobiele webversie kan de toegankelijkheid verder vergroten.
\end{itemize}

\section{Praktijkervaring en gebruikersfeedback}
\label{sec:casus-feedback}

Tijdens de testfase werd de casus in een niet echte omgeving uitgevoerd door een testteam. De gebruikers gaven aan dat zij sneller konden schakelen tussen taken, minder tijd kwijt waren aan het zoeken naar informatie en dat de communicatie met de klant soepeler verliep. Vooral de centrale resource library en de klantenzone werden als grote voordelen ervaren. Enkele testers gaven aan dat zij in de toekomst graag nog meer automatisering en integraties zouden zien, evenals een mobiele app.

\section{Conclusie van de casusvergelijking}
\label{sec:casus-conclusie}

De casus toont aan dat het ontwikkelde platform een duidelijke meerwaarde biedt ten opzichte van het werken met een versnipperd landschap van losse tools. Door alle kernfunctionaliteiten te bundelen in één omgeving, wordt de workflow van web agencies efficiënter, transparanter en gebruiksvriendelijker. Verdere optimalisatie en uitbreiding van integraties kunnen de impact van het platform in de praktijk nog verder vergroten. 
\chapter{Design en Architectuur}
\label{ch:design}

Dit hoofdstuk beschrijft in detail de design- en architectuurfase van het platform, inclusief de technische keuzes, ontwerpbeslissingen en de onderliggende motieven. Het doel is om een helder en volledig beeld te geven van hoe het platform is opgebouwd, waarom bepaalde keuzes zijn gemaakt, en hoe de verschillende componenten samenwerken.

\section{Technische Architectuur}
\label{sec:tech-architectuur}

Het platform is opgezet als een moderne webapplicatie, waarbij gebruik is gemaakt van een fullstack benadering met Next.js als kern. De architectuur is modulair en schaalbaar opgezet, zodat toekomstige uitbreidingen eenvoudig kunnen worden doorgevoerd.

\subsection{Overzicht van de architectuur}
De applicatie bestaat uit de volgende hoofdcomponenten:
\begin{itemize}
    \item \textbf{Frontend}: Gebouwd met Next.js en React, met ondersteuning voor server-side rendering (SSR) en statische generatie (SSG) waar mogelijk.
    \item \textbf{Backend}: API-routes in Next.js, die communiceren met Supabase en Prisma voor dataopslag en authenticatie.
    \item \textbf{Database}: Supabase (PostgreSQL) als primaire database, beheerd via Prisma ORM.
    \item \textbf{Authenticatie}: Supabase Auth en Better Auth voor gebruikersbeheer, teambeheer en toegangscontrole.
    \item \textbf{Externe integraties}: Google Calendar API voor synchronisatie van taken, Figma API voor design assets, en e-mailnotificaties.
    \item \textbf{Realtime functionaliteit}: Supabase Realtime voor live updates van taken, notificaties en resource library.
\end{itemize}

\subsection{Architectuurdiagram}
\begin{figure}[h]
    \centering
    %\includegraphics[width=0.9\textwidth]{architectuur-overzicht.png}
    \caption{Globaal overzicht van de architectuur van het platform}
    \label{fig:architectuur-overzicht}
\end{figure}

\section{Frontend Architectuur}
\label{sec:frontend-architectuur}

De frontend is opgebouwd met Next.js (App Router) en React. Er is gekozen voor een component-gebaseerde structuur, waarbij herbruikbaarheid en onderhoudbaarheid centraal staan.

\subsection{Structuur en routing}
\begin{itemize}
    \item \textbf{App Router}: Alle routes zijn opgezet via de Next.js App Router, wat zorgt voor duidelijke scheiding tussen publieke, team- en klantenzones.
    \item \textbf{Layout}: Gebruik van shared layouts voor consistente navigatie, theming en toegangscontrole.
    \item \textbf{Code splitting}: Dynamische imports en React.lazy voor het optimaliseren van de laadtijd.
    \item \textbf{State management}: Gebruik van React Context en (optioneel) Zustand of Redux voor globale state, zoals gebruikersdata, notificaties en teamselectie.
    \item \textbf{Theming}: Implementatie van een dark/light theme via CSS-variabelen en context.
\end{itemize}

\subsection{Belangrijke UI-componenten}
\begin{itemize}
    \item \textbf{Dashboard}: Overzicht van projecten, taken, notificaties en teamactiviteiten.
    \item \textbf{Project board}: Kanban-achtige weergave van taken per projectfase, met drag-and-drop via @dnd-kit.
    \item \textbf{Resource library}: Tab-based interface voor code, design en documenten, met Monaco Editor en rich text editor.
    \item \textbf{Klantenzone}: Afgeschermde omgeving voor klanten, met beperkte toegang tot relevante projecten en documenten.
    \item \textbf{Notificatiecentrum}: Live updates van taken, opmerkingen en statuswijzigingen.
    \item \textbf{Instellingen}: Beheer van teams, gebruikers, rollen en integraties.
\end{itemize}

\subsection{Toegankelijkheid en UX}
\begin{itemize}
    \item \textbf{Responsief ontwerp}: Volledige ondersteuning voor desktop, tablet en mobiel.
    \item \textbf{Toegankelijkheid}: Gebruik van ARIA-labels, toetsenbordnavigatie en kleurcontrasten.
    \item \textbf{Performance}: Lazy loading, debouncing en optimalisatie van zware componenten zoals de Monaco Editor.
\end{itemize}

\section{Backend Architectuur}
\label{sec:backend-architectuur}

De backend bestaat uit Next.js API-routes, die als serverless functies draaien. Deze routes verzorgen de communicatie met Supabase, Prisma en externe APIs.

\subsection{API-structuur}
\begin{itemize}
    \item \textbf{RESTful endpoints}: Voor CRUD-operaties op projecten, taken, teams, resources en notificaties.
    \item \textbf{Authenticatie middleware}: Controleert JWT-tokens van Supabase/Better Auth en bepaalt toegang op basis van rol en team.
    \item \textbf{Error handling}: Gebruik van error boundaries en centrale error handlers voor consistente foutafhandeling.
    \item \textbf{Rate limiting}: Bescherming tegen misbruik van API-routes.
\end{itemize}

\subsection{Databaseontwerp}
\begin{itemize}
    \item \textbf{Relaties}: Tabellen voor gebruikers, teams, projecten, taken, resources, notificaties, rollen en klantenzones.
    \item \textbf{Prisma schema}: Gebruik van Prisma Migrate voor versiebeheer van het datamodel.
    \item \textbf{Connection pooling}: Geconfigureerd voor productieomgevingen via Supabase.
    \item \textbf{Transacties}: Voor kritieke operaties zoals bulk edits en statuswijzigingen.
\end{itemize}

\subsection{Realtime functionaliteit}
\begin{itemize}
    \item \textbf{Supabase Realtime}: Live updates voor taken, notificaties en resource library.
    \item \textbf{Websockets}: Voor directe communicatie tussen frontend en backend bij kritieke updates.
\end{itemize}

\section{Authenticatie, Autorisatie en Teambeheer}
\label{sec:auth-teams}

\subsection{Authenticatie}
\begin{itemize}
    \item \textbf{Supabase Auth}: Voor standaard gebruikersauthenticatie (e-mail, wachtwoord, magic links).
    \item \textbf{Better Auth}: Voor geavanceerde authenticatie, zoals OAuth, 2FA en single sign-on.
\end{itemize}

\subsection{Teambeheer en toegangscontrole}
\begin{itemize}
    \item Gebruikers kunnen teams aanmaken, uitnodigen en beheren.
    \item Elk team heeft eigen projecten, resource library en klantenzone.
    \item Rollen bepalen de toegangsrechten (admin, medewerker, klant).
    \item Toegangscontrole op API-niveau en in de frontend.
\end{itemize}

\subsection{Klantenzone}
\begin{itemize}
    \item Per team kan een klantenzone worden geactiveerd.
    \item Klanten zien enkel hun eigen projecten, documenten en voortgang.
    \item Mogelijkheid tot aanvragen van revisies en feedback.
\end{itemize}

\section{Notificatiesysteem}
\label{sec:notificaties}

Het notificatiesysteem is ontworpen om gebruikers continu op de hoogte te houden van relevante gebeurtenissen.

\begin{itemize}
    \item \textbf{Triggers}: Nieuwe taken, statuswijzigingen, opmerkingen, resource updates, teamactiviteiten.
    \item \textbf{Kanalen}: In-app notificaties, optioneel e-mail.
    \item \textbf{Realtime}: Directe updates via Supabase Realtime.
    \item \textbf{Instelbaarheid}: Gebruikers kunnen voorkeuren instellen voor welke meldingen ze willen ontvangen.
\end{itemize}

\section{Project- en Task Management}
\label{sec:project-task}

\begin{itemize}
    \item \textbf{Projectfases}: Elk project kan worden opgedeeld in fases, die visueel worden weergegeven.
    \item \textbf{Taken}: Taken kunnen worden toegewezen, voorzien van deadlines, gekoppeld aan Google Calendar, en in bulk worden aangepast.
    \item \textbf{Templates}: Voor veelvoorkomende taken en documenten kunnen templates worden aangemaakt en hergebruikt.
    \item \textbf{Views}: Zowel persoonlijke als team views voor het filteren en prioriteren van taken.
    \item \textbf{Inline editing}: Status en details van taken kunnen direct vanuit het overzicht worden aangepast.
\end{itemize}

\section{Resource Library}
\label{sec:resource-library}

\begin{itemize}
    \item \textbf{Code snippets}: Monaco Editor voor bewerken, syntax highlighting en categorisatie.
    \item \textbf{Design assets}: Integratie met Figma, upload van afbeeldingen, beheer van kleurenpaletten.
    \item \textbf{Documenten}: Rich text editor, versiebeheer, templates.
    \item \textbf{Bulk acties}: Multi-edit en verplaatsing van meerdere items tegelijk.
    \item \textbf{Toegangscontrole}: Enkel teamleden met de juiste rechten kunnen resources aanpassen.
\end{itemize}

\section{Integraties met Externe Diensten}
\label{sec:integraties}

\begin{itemize}
    \item \textbf{Google Calendar API}: Synchronisatie van deadlines en meetings.
    \item \textbf{Figma API}: Importeren van design assets en live previews.
    \item \textbf{E-mail}: Versturen van notificaties en uitnodigingen.
   
\end{itemize}

\section{Beveiliging en Privacy}
\label{sec:security}

\begin{itemize}
    \item \textbf{Authenticatie en autorisatie}: JWT-tokens, rolgebaseerde toegang, 2FA.
    \item \textbf{Data encryptie}: Gevoelige data wordt versleuteld opgeslagen.
    \item \textbf{GDPR}: Dataopslag en verwerking conform GDPR-richtlijnen.
    \item \textbf{Input validatie}: Zowel client- als server-side.
    \item \textbf{Logging en monitoring}: Voor detectie van verdachte activiteiten.
   
\end{itemize}

\section{Schaalbaarheid en Performantie}
\label{sec:schaalbaarheid}

\begin{itemize}
    \item \textbf{Serverless deployment}: Next.js API-routes draaien als serverless functies.
    \item \textbf{Connection pooling}: Voor efficiënte databaseconnecties.
    \item \textbf{Caching}: Gebruik van ISR (Incremental Static Regeneration) en client-side caching.
    \item \textbf{Code splitting en lazy loading}: Voor snelle laadtijden.
    \item \textbf{Monitoring}: Integratie met tools zoals Vercel Analytics of Sentry.
\end{itemize}

\section{Designbeslissingen en Overwegingen}
\label{sec:design-keuzes}

\begin{itemize}
    \item \textbf{Modulariteit}: Componenten en modules zijn los van elkaar te ontwikkelen en te testen.
    \item \textbf{Herbruikbaarheid}: Templates, componenten en logica zijn generiek opgezet.
    \item \textbf{Iteratief ontwerp}: Regelmatige feedbackrondes en bijsturing op basis van gebruikerservaring.
    \item \textbf{Toekomstbestendigheid}: Architectuur is voorbereid op toekomstige uitbreidingen (bv. mobiele app, extra integraties).
\end{itemize}

\section{Voorbeeld code en diagrammen}
\label{sec:code-diagrammen}

\begin{itemize}
    \item \textbf{Voorbeeld Next.js API route}: Zie methodologiehoofdstuk.
    \item \textbf{Diagrammen}: Zie figuren in dit hoofdstuk voor architectuuroverzicht en datamodellen.
\end{itemize}

\section{Samenvatting}
\label{sec:samenvatting}

De gekozen design en architectuur zorgen voor een schaalbaar, veilig en gebruiksvriendelijk platform dat eenvoudig kan worden uitgebreid. Door te kiezen voor moderne technologieën en een modulaire opzet, is het platform voorbereid op de noden van web agencies en hun klanten.

% Vragen voor jou:
% - Zijn er nog andere externe integraties gebruikt (bv. Slack, Jira, ...)? 
% - Zijn er specifieke security audits of pentests uitgevoerd?
% - Wil je extra diagrammen of codevoorbeelden toevoegen? (Lever gerust aan!)
% - Zijn er specifieke design patterns of architecturale principes die je extra wil toelichten?
\chapter{Ontwikkeling en Implementatie}
\label{ch:ontwikkeling}

Dit hoofdstuk behandelt het volledige ontwikkelingsproces van het platform, met een diepgaande focus op de implementatie van de kernfunctionaliteiten, de gebruikte technieken, de iteratieve aanpak en de uitdagingen die tijdens de ontwikkeling zijn overwonnen. Het doel is om een transparant en volledig beeld te geven van hoe het platform tot stand is gekomen, van eerste setup tot de uiteindelijke oplevering.

\section{Projectopzet en Voorbereiding}
\label{sec:projectopzet}

De ontwikkeling startte met het opzetten van een moderne ontwikkelomgeving. Er werd gekozen voor een standaard \textbf{Next.js} projectstructuur, waarbij frontend en backend code in één repository worden beheerd. De belangrijkste stappen in deze fase waren:
\begin{itemize}
    \item Initialisatie van een Next.js project met Typescript voor typeveiligheid en schaalbaarheid.
    \item Integratie van Supabase als backend-as-a-service voor database, authenticatie en realtime functionaliteit.
    \item Installatie en configuratie van Prisma als ORM voor het modelleren van de database en het uitvoeren van migraties.
    \item Opzetten van een gestructureerde mappenstructuur voor componenten, pagina's, API-routes en util-functies.
    \item Inrichten van ESLint en Prettier voor consistente codekwaliteit.
    \item Gebruik van Git en GitHub voor versiebeheer en samenwerking.
\end{itemize}

\section{Iteratief Ontwikkelingsproces}
\label{sec:iteratief-proces}

De ontwikkeling verliep volgens een iteratief proces, waarbij telkens kleine deelmodules werden uitgewerkt, getest en bijgestuurd op basis van feedback. De belangrijkste iteraties waren:
\begin{itemize}
    \item MVP (Minimum Viable Product): Basisfunctionaliteiten zoals authenticatie, teambeheer, project- en takenbeheer.
    \item Feature-uitbreidingen: Toevoegen van resource library, klantenzone, notificatiesysteem en integraties.
    \item UX/UI-verbeteringen: Optimalisatie van de gebruikerservaring op basis van feedback van testgebruikers.
    \item Performance en security: Verbeteren van laadtijden, beveiliging en schaalbaarheid.
\end{itemize}

\section{Implementatie van Kernfunctionaliteiten}
\label{sec:implementatie-kern}

\subsection{Authenticatie en Teambeheer}
\begin{itemize}
    \item Implementatie van gebruikersregistratie en login via Supabase Auth en Better Auth.
    \item Gebruikers kunnen teams aanmaken, beheren en leden uitnodigen.
    \item Uitnodigingen voor teams verlopen via een unieke invite-link die gedeeld kan worden met potentiële teamleden. Daarnaast is het mogelijk om uitnodigingen per e-mail te versturen.
    \item Rolgebaseerde toegang: admins, medewerkers en klanten met verschillende rechten.
    \item Toewijzing van projecten, taken en resources aan specifieke teams.
\end{itemize}

\subsection{Project- en Task Management}
\begin{itemize}
    \item CRUD-operaties voor projecten en taken via Next.js API-routes en Prisma.
    \item Taken kunnen worden toegewezen aan teamleden, voorzien van deadlines en gekoppeld aan projectfases.
    \item Drag-and-drop interface voor het plannen en herschikken van taken, gerealiseerd met @dnd-kit.
    \item Inline editing van taakstatussen en bulkbewerking van meerdere taken tegelijk.
    \item Templates voor veelvoorkomende taken en workflows.
    \item Synchronisatie van deadlines met Google Calendar API.
    \item Voor het ophalen en muteren van taken zijn custom React hooks ontwikkeld die volgens een local-first principe werken: wijzigingen worden eerst lokaal doorgevoerd voor directe feedback aan de gebruiker, en vervolgens asynchroon gesynchroniseerd met de server. Dit zorgt voor een zeer snelle gebruikerservaring.
\end{itemize}

\subsection{Resource Library}
\begin{itemize}
    \item Centrale opslagplaats per team voor code snippets, design assets en documenten.
    \item Integratie van Monaco Editor voor het bewerken van codefragmenten met syntax highlighting.
    \item Upload en beheer van design assets (afbeeldingen, Figma links, kleurenpaletten).
    \item Voor het uploaden en optimaliseren van afbeeldingen wordt gebruik gemaakt van Cloudflare Images.
    \item Voor projectdocumentatie is gekozen voor de \textbf{Slate} rich text editor, vanwege de flexibiliteit en uitbreidbaarheid.
    \item Documenten kunnen als template worden opgeslagen voor hergebruik. Versiebeheer van documenten is geïmplementeerd door bij elke wijziging een nieuwe versie op te slaan in de database, zodat eerdere versies kunnen worden teruggezet indien nodig.
    \item Multi-edit functionaliteit voor het snel aanpassen of verplaatsen van meerdere library items.
\end{itemize}

\subsection{Klantenzone}
\begin{itemize}
    \item Per team een afgeschermde klantenzone, waar klanten enkel hun eigen projecten en documenten kunnen zien.
    \item Mogelijkheid voor klanten om feedback te geven, revisies aan te vragen en de voortgang te volgen.
    \item Beheer van klanttoegang via rollen en toegangscontrole.
\end{itemize}

\subsection{Notificatiesysteem}
\begin{itemize}
    \item Realtime notificaties bij updates aan taken, projecten, opmerkingen en resource library.
    \item In-app notificatiecentrum en optioneel e-mailnotificaties.
    \item Instelbare voorkeuren per gebruiker voor het ontvangen van meldingen.
    \item Gebruik van Supabase Realtime voor directe updates.
\end{itemize}

\subsection{Persoonlijke en Team Views}
\begin{itemize}
    \item Gebruikers kunnen persoonlijke views aanmaken voor hun eigen taken en prioriteiten.
    \item Team views tonen het volledige overzicht van alle taken, projecten en resources binnen het team.
    \item Filters en sorteeropties voor snelle toegang tot relevante informatie.
\end{itemize}

\subsection{Fasebeheer en Inline Editing}
\begin{itemize}
    \item Projecten kunnen worden opgedeeld in fases, die visueel worden weergegeven op het project board.
    \item Inline editing van project- en taakstatussen direct vanuit het overzicht.
    \item Bulkacties voor het snel aanpassen van meerdere items.
\end{itemize}

\section{Integratie van Externe Diensten}
\label{sec:integraties}

\begin{itemize}
    \item Google Calendar API: Synchronisatie van deadlines en meetings met de persoonlijke agenda van gebruikers.
    \item Figma API: Importeren van design assets en live previews in de resource library.
    \item Cloudflare Images: Voor het uploaden, optimaliseren en serveren van afbeeldingen binnen het platform.
    \item E-mail: Versturen van uitnodigingen, notificaties en updates.
\end{itemize}

\section{Testing en Debugging}
\label{sec:testing-debugging}

\begin{itemize}
    \item Handmatige en geautomatiseerde tests van alle kernfunctionaliteiten.
    \item Gebruik van Jest en React Testing Library voor unit- en integratietests.
    \item Usability tests met eindgebruikers (collega-studenten, docenten, of mensen uit het werkveld).
    \item Debugging met browser developer tools, logging en monitoring via Vercel Analytics of Sentry.
    \item Security checks op authenticatie, toegangscontrole en dataopslag.
    % Voor end-to-end testing en security audits zijn (nog) geen specifieke tools of processen ingezet.
\end{itemize}

\section{Uitdagingen en Oplossingen}
\label{sec:uitdagingen}

Tijdens de ontwikkeling zijn verschillende uitdagingen overwonnen, waaronder:
\begin{itemize}
    \item \textbf{Rechten en toegangscontrole}: Het correct afschermen van data tussen teams en klanten vergde een zorgvuldig opgezet autorisatiemodel.
    \item \textbf{Realtime updates}: Het implementeren van realtime functionaliteit zonder performanceverlies.
    \item \textbf{Bulkacties en inline editing}: Het efficiënt verwerken van bulk updates en het direct aanpassen van data in de UI.
    \item \textbf{Integratie van meerdere externe APIs}: Het combineren van verschillende externe diensten op een consistente manier.
    \item \textbf{Schaalbaarheid}: Zorgen dat het platform performant blijft bij groeiend gebruik.
\end{itemize}

\section{Refactoring en Optimalisatie}
\label{sec:refactoring}

Gedurende het project zijn verschillende refactorings doorgevoerd:
\begin{itemize}
    \item Herstructurering van componenten voor betere herbruikbaarheid.
    \item Optimalisatie van database queries met Prisma voor snellere responstijden.
    \item Gebruik van React.memo en useMemo voor performance optimalisatie in de frontend.
    \item Verbetering van error handling en logging.
\end{itemize}

\section{Overzicht van de Workflow}
\label{sec:workflow}

Het ontwikkelproces werd gekenmerkt door een iteratieve aanpak, waarbij telkens werd teruggekoppeld naar de centrale requirements en praktijknoden. Door te werken in korte sprints en regelmatig feedback te verzamelen, kon het platform stapsgewijs worden opgebouwd en bijgestuurd waar nodig.

\section{Voorbeeld codefragmenten}
\label{sec:codefragmenten}

Hieronder volgt een voorbeeld van een Next.js API route voor het aanmaken van een taak (zie methodologiehoofdstuk voor meer voorbeelden):

\begin{listing}[H]
\begin{minted}{typescript}
// Voorbeeld: Next.js API route voor het aanmaken van een taak
import { supabase } from '../../lib/supabaseClient';

export default async function handler(req, res) {
  if (req.method === 'POST') {
    const { title, projectId, assignedTo, dueDate } = req.body;
    const { data, error } = await supabase
      .from('tasks')
      .insert([{ title, project_id: projectId, assigned_to: assignedTo, due_date: dueDate }]);
    if (error) return res.status(400).json({ error: error.message });
    res.status(200).json({ task: data });
  }
}
\end{minted}
\caption{Voorbeeld van een API route voor het aanmaken van een taak}
\end{listing}

% Voeg gerust meer codevoorbeelden toe voor belangrijke onderdelen zoals authenticatie, resource upload, of integratie met externe APIs.

\section{Samenvatting}
\label{sec:samenvatting}

De ontwikkeling van het platform verliep via een gestructureerd, iteratief proces met veel aandacht voor schaalbaarheid, gebruiksvriendelijkheid en veiligheid. Door moderne technologieën en best practices te combineren, is een solide basis gelegd voor een toekomstbestendig platform voor web agencies.


\chapter{Testing en Evaluatie}
\label{ch:testing}

Dit hoofdstuk beschrijft de verschillende testfasen en evaluatiemethoden die zijn gebruikt om de kwaliteit en effectiviteit van het platform te waarborgen.

\section{Test Methodologie}
\label{sec:test-methodologie}

details over de implementatie volgen in dit hoofdstuk...
\chapter{Analyse en Resultaten}
\label{ch:analyse}

Dit hoofdstuk presenteert de analyse van de testresultaten en de bevindingen uit de evaluatiefase.

\section{Analyse van Testresultaten}
\label{sec:analyse-resultaten}

details over de implementatie volgen in dit hoofdstuk...
%%=============================================================================
%% Conclusie
%%=============================================================================

\chapter{Conclusie}%
\label{ch:conclusie}

Deze bachelorproef had als doel het ontwikkelen van een geïntegreerd platform dat de workflows van web agencies optimaliseert door projectmanagement, resourcebeheer en klantcommunicatie te combineren in één samenhangend systeem. In deze conclusie worden de belangrijkste bevindingen, de bijdrage aan het vakgebied, de relevantie voor de praktijk en de aanbevelingen voor toekomstig onderzoek samengevat.

\section{Antwoord op de Onderzoeksvraag}

De centrale onderzoeksvraag luidde: \emph{Hoe kan een geïntegreerd platform worden ontwikkeld dat de workflows van web agencies optimaliseert door project management, resource beheer en klantcommunicatie te combineren in één samenhangend systeem?}

Uit de literatuurstudie (\autoref{ch:stand-van-zaken}) bleek dat web agencies in de praktijk geconfronteerd worden met gefragmenteerde workflows, inefficiënt resourcebeheer en beperkte klanttransparantie. Bestaande oplossingen zoals Jira, Trello, Figma en Notion bieden elk slechts een deel van de benodigde functionaliteit en vereisen vaak het schakelen tussen verschillende tools, wat leidt tot tijdsverlies en miscommunicatie.

Het ontwikkelde platform biedt een antwoord op deze problemen door:
\begin{itemize}
    \item \textbf{Centrale toegang} tot projectmanagement, resource library en klantenzone binnen één applicatie.
    \item \textbf{Local-first synchronisatie} en custom hooks, waardoor gebruikers direct feedback krijgen en sneller kunnen werken.
    \item \textbf{Uitgebreide resource library} met ondersteuning voor code (Monaco Editor), design assets (Figma-integratie, Cloudflare Images) en documenten (Slate rich text editor met versiebeheer).
    \item \textbf{Team- en klantenzones} die privacy en overzicht waarborgen.
    \item \textbf{Realtime notificaties} via Supabase Realtime, zodat gebruikers altijd op de hoogte zijn van relevante wijzigingen.
    \item \textbf{Naadloze integraties} met Google Calendar, Figma en Cloudflare Images.
    \item \textbf{Bulkacties, inline editing en templates} voor efficiënt beheer van taken en resources.
\end{itemize}

\section{Bijdrage aan het Onderzoeksdomein en het Werkveld}

Deze bachelorproef levert een concrete bijdrage aan het onderzoeksdomein van digitale workflow-optimalisatie voor web agencies. Door de combinatie van literatuurstudie, analyse van bestaande oplossingen en de ontwikkeling van een proof-of-concept, wordt aangetoond dat een geïntegreerd platform daadwerkelijk kan leiden tot efficiëntere werkprocessen, betere samenwerking en verhoogde klanttevredenheid.

Voor het werkveld biedt het platform een schaalbare en toekomstbestendige oplossing die inspeelt op de actuele noden van web agencies:
\begin{itemize}
    \item Minder tijdverlies door het elimineren van context-switching tussen tools.
    \item Betere samenwerking door centrale opslag en versiebeheer van alle projectresources.
    \item Meer transparantie en betrokkenheid van klanten via een afgeschermde klantenzone.
    \item Snellere onboarding van nieuwe teamleden dankzij templates en duidelijke workflowstructuren.
\end{itemize}

\section{Kritische Reflectie op het Resultaat}

Hoewel het platform in hoge mate voldoet aan de vooraf opgestelde requirements, zijn er ook enkele beperkingen en aandachtspunten:
\begin{itemize}
    \item De evaluatie is voornamelijk gebaseerd op kwalitatieve feedback van een beperkte groep gebruikers. Grootschalige, kwantitatieve studies zijn nodig om de impact op productiviteit en klanttevredenheid objectief te meten.
    \item Er zijn nog geen geautomatiseerde end-to-end tests of security audits uitgevoerd. Voor productiegebruik is verdere validatie noodzakelijk.
    \item Integraties met andere populaire tools zoals Slack, Jira of cloud storage zijn nog niet geïmplementeerd, maar worden door gebruikers wel gevraagd.
    \item AI-ondersteunde features, zoals slimme taakverdeling of automatische resource tagging, zijn nog niet aanwezig maar vormen een interessante piste voor toekomstig onderzoek.
\end{itemize}

\section{Aanbevelingen en Toekomstperspectief}

Op basis van de analyse en de ontvangen feedback worden de volgende aanbevelingen gedaan:
\begin{itemize}
    \item \textbf{Uitbreiden van integraties} met andere tools om het platform nog waardevoller te maken voor verschillende soorten web agencies.
    \item \textbf{Implementeren van AI-functionaliteit} voor automatisering, slimme suggesties en verdere workflow-optimalisatie.
    \item \textbf{Opzetten van een geautomatiseerd test- en auditproces} om de betrouwbaarheid en veiligheid van het platform te waarborgen.
    \item \textbf{Uitvoeren van grootschalige gebruikersstudies} om de effectiviteit van het platform in diverse praktijkomgevingen te valideren.
    \item \textbf{Optimaliseren voor mobiele apparaten} zodat gebruikers altijd en overal toegang hebben tot hun projecten en resources.
\end{itemize}

\section{Reflectie en Slotbeschouwing}

De uitkomst van deze bachelorproef bevestigt de hypothese dat een geïntegreerd platform een significante meerwaarde kan bieden voor web agencies. De combinatie van projectmanagement, resourcebeheer en klantcommunicatie in één systeem resulteert in efficiëntere processen, betere samenwerking en meer tevreden klanten. De gekozen technologieën (Next.js, Supabase, Prisma, React, Monaco Editor, Slate, Cloudflare Images) en de modulaire architectuur zorgen ervoor dat het platform eenvoudig kan worden uitgebreid en aangepast aan toekomstige noden.

Hoewel het project nog ruimte voor groei en optimalisatie biedt, vormt het een solide basis voor verdere innovatie in het domein van digitale samenwerkingstools voor web agencies. De bachelorproef levert niet alleen een werkend proof-of-concept op, maar ook een blauwdruk voor hoe moderne web agencies hun workflows kunnen stroomlijnen en hun concurrentiepositie kunnen versterken.




%---------- Bijlagen -----------------------------------------------------------

\appendix

\chapter{Onderzoeksvoorstel}

Het onderwerp van deze bachelorproef is gebaseerd op een onderzoeksvoorstel dat vooraf werd beoordeeld door de promotor. Dat voorstel is opgenomen in deze bijlage.

%% TODO: 
%\section*{Samenvatting}

% Kopieer en plak hier de samenvatting (abstract) van je onderzoeksvoorstel.

% Verwijzing naar het bestand met de inhoud van het onderzoeksvoorstel
%---------- Inleiding ---------------------------------------------------------

% TODO: Is dit voorstel gebaseerd op een paper van Research Methods die je
% vorig jaar hebt ingediend? Heb je daarbij eventueel samengewerkt met een
% andere student?
% Zo ja, haal dan de tekst hieronder uit commentaar en pas aan.

%\paragraph{Opmerking}

% Dit voorstel is gebaseerd op het onderzoeksvoorstel dat werd geschreven in het
% kader van het vak Research Methods dat ik (vorig/dit) academiejaar heb
% uitgewerkt (met medesturent VOORNAAM NAAM als mede-auteur).
% 

\section{Inleiding}%
\label{sec:inleiding}

Veel web agencies gebruiken momenteel verschillende losse tools voor het beheren van projecten, communicatie, en resources. Denk bijvoorbeeld aan tools zoals Trello voor takenbeheer, Slack voor communicatie, en Figma voor design. Hoewel deze tools nuttig zijn, ontbreekt het vaak aan een centrale plek waar alles samenkomt. Dit leidt vaak tot inefficiënties, miscommunicatie en frustratie bij zowel medewerkers als klanten. Deze knelpunten komen voort uit het gebruik van verschillende systemen die niet altijd goed integreren, wat leidt tot extra werk, vertragingen en misverstanden.
\\
\\
Dit onderzoeksproject heeft als doel een all-in-one platform te ontwikkelen dat specifiek gericht is op de behoeften van web agencies. Het platform biedt een geïntegreerde oplossing die alle noodzakelijke tools bevat. Naast standaardfunctionaliteiten zoals project- en takenbeheer, bevat het platform een unieke resource library waar medewerkers designs, Figma-bestanden en code snippets kunnen opslaan en live aanpassen. Deze mogelijkheid vereenvoudigt het samenwerken tussen ontwerpers en ontwikkelaars aanzienlijk, wat de efficiëntie zou kunnen verbeteren.
\\
\\
Daarnaast is er een klantenzone, waar klanten toegang krijgen tot een overzicht van alle deliverables. Ze kunnen hier precies zien wat er wanneer gebeurt en voltooide deliverables bekijken of downloaden. Als een klant herhalingen nodig heeft van bepaalde onderdelen, kan dit eenvoudig worden aangevraagd via het platform. Dit verhoogt de transparantie en klanttevredenheid.
\\
\\
De centrale probleemstelling is: \textcolor{gray}{Hoe kan een gecentraliseerd platform de interne workflow van web agencies verbeteren en tegelijkertijd de transparantie naar klanten vergroten?}


\subsection{Deelvragen m.b.t. het probleemdomein}

De deelvragen met betrekking tot het probleemdomein zijn gericht op het identificeren van de specifieke bottlenecks in de processen en communicatie binnen web agencies. Deze deelvragen richten zich op het begrijpen van de interne werkstromen en de knelpunten die ontstaan door het gebruik van verschillende losse tools. Enkele mogelijke deelvragen zijn:

\begin{itemize}
    \item \textcolor{gray}{Wat zijn de specifieke bottlenecks die ontstaan door het gebruik van gescheiden tools voor projectbeheer, communicatie en resourcebeheer binnen web agencies?}
    \item \textcolor{gray}{Hoe beïnvloeden de gescheiden systemen de samenwerking tussen ontwerpers, ontwikkelaars en klanten? Welke communicatieproblemen ontstaan hierdoor?}
    \item \textcolor{gray}{In welke mate zorgen de integratieproblemen van tools voor vertragingen in het projectproces en onduidelijkheden in de klantcommunicatie?}
    \item \textcolor{gray}{Wat zijn de uitdagingen bij het beheren van design- en code-resources in een gefragmenteerde werkomgeving, en hoe beïnvloedt dit de productiviteit van medewerkers?}
    \item \textcolor{gray}{Hoe beïnvloedt het gebrek aan transparantie en overzicht voor klanten de klanttevredenheid en hun vertrouwen in het projectproces?}
\end{itemize}

\subsection{Deelvragen m.b.t. het oplossingsdomein}

De deelvragen met betrekking tot het oplossingsdomein richten zich op de kenmerken en vereisten van een platform dat deze bottlenecks en inefficiënties kan verhelpen. Deze vragen helpen om de belangrijkste functionaliteiten van het platform te bepalen die de workflow binnen web agencies kunnen verbeteren en tegelijkertijd de klanttevredenheid kunnen verhogen:

\begin{itemize}
    \item \textcolor{gray}{Welke functionaliteiten moeten er aanwezig zijn in een platform om een geïntegreerde oplossing te bieden voor projectbeheer, communicatie en resourcebeheer binnen web agencies?}
    \item \textcolor{gray}{Hoe kan een platform de samenwerking tussen ontwerpers en ontwikkelaars verbeteren door middel van een resource library voor live bewerking van Figma-bestanden en code snippets?}
    \item \textcolor{gray}{Welke elementen moeten er aanwezig zijn in een klantenzone om de transparantie en klanttevredenheid te verbeteren?}
    \item \textcolor{gray}{Hoe kan real-time samenwerking tussen medewerkers en klanten effectief worden geïmplementeerd in een platform?}
\end{itemize}

Het einddoel van dit project is een werkend prototype dat getest kan worden door agencies en klanten, met als doel de werkprocessen te verbeteren, de samenwerking te vergemakkelijken en de communicatie met klanten te optimaliseren.

%---------- Stand van zaken ---------------------------------------------------

\section{Literatuurstudie}%
\label{sec:literatuurstudie}
Er zijn verschillende tools beschikbaar voor projectbeheer en communicatie binnen web agencies, zoals Trello, Asana en Slack, maar geen van deze biedt een oplossing die alle functies samenbrengt. Veel van deze platforms bieden geen mogelijkheid om code snippets live te bewerken of Figma-bestanden op een georganiseerde manier te delen. Onderzoek van~\textcite{Reid2014} en~\textcite{Alexander2019} toont aan dat de samenwerking tussen ontwerpers en ontwikkelaars vaak moeilijker is door het gebruik van verschillende, niet-geïntegreerde tools.
\\
\\
Er is een duidelijke vraag naar een centrale plek waar verschillende functies voor projectbeheer, communicatie en resources samenkomen.~\textcite{Santoso2024} stelt dat zulke platforms niet alleen projectbeheer en taakorganisatie moeten bieden, maar ook samenwerking tussen verschillende teams, zoals ontwerpers en ontwikkelaars, moeten bevorderen. Het combineren van projectbeheer, klantcommunicatie en resourcebeheer in één systeem kan volgens~\textcite{Chris2024} de klanttevredenheid verbeteren en de werkdruk bij medewerkers verlichten.
\\
\\
Dit platform richt zich op het oplossen van deze problemen door een alles-in-één oplossing te bieden voor projectbeheer en resourcebeheer, met de resource library als een belangrijk onderdeel van de workflow. Er is weinig onderzoek dat de integratie van projectmanagement, klantcommunicatie en een geavanceerde resource library bespreekt. Dit onderzoek zal deze nieuwe benadering verder onderzoeken.


% Voor literatuurverwijzingen zijn er twee belangrijke commando's:
% \autocite{KEY} => (Auteur, jaartal) Gebruik dit als de naam van de auteur
%   geen onderdeel is van de zin.
% \textcite{KEY} => Auteur (jaartal)  Gebruik dit als de auteursnaam wel een
%   functie heeft in de zin (bv. ``Uit onderzoek door Doll & Hill (1954) bleek
%   ...'')


%---------- Methodologie ------------------------------------------------------
\section{Methodologie}%
\label{sec:methodologie}

De ontwikkelingsmethode zal bestaan uit 3 fases, waarbij het platform in fasen wordt ontworpen, getest en verbeterd op basis van feedback van school.
\\
\\
In de eerste fase zal het ontwerp van de gebruikersinterface(UI) en gebruikerservaring(UX) plaatsvinden, waarbij de nadruk ligt op gebruiksvriendelijkheid en functionaliteit. In de tweede fase zal het platform worden ontwikkeld in Next.js voor de webversie. Supabase of neon zal worden gebruikt voor de opslag van gegevens, en Socket.io zal worden ingezet voor real-time samenwerking. Tijdens de derde fase worden kwalitatieve interviews en surveys uitgevoerd met medewerkers van web agencies om te achterhalen in hoeverre het platform hun werkprocessen verbeterd. Ook worden A/B-tests uitgevoerd om de effectiviteit van specifieke functionaliteiten te evalueren.
\\
\\
De tools die gebruikt zullen worden, zijn onder andere Figma voor het ontwerp, Visual Studio Code voor de ontwikkeling, en GitHub voor versiebeheer. De tijdsplanning is als volgt:

\begin{itemize}
    \item Fase 1 (UI/UX ontwerp): 1 week
    \item Fase 2 (Platformontwikkeling): 9 weken
    \item Fase 3 (Testen en feedback): 2 weken
    
\end{itemize}

%---------- Verwachte resultaten ----------------------------------------------
\section{Verwacht resultaat, conclusie}%
\label{sec:verwachte_resultaten}

Het verwachte resultaat is een werkend platform dat takenbeheer, meetings, resourcebeheer, en klantinteractie combineert in één systeem. Het biedt een gestroomlijnde ervaring voor zowel medewerkers als klanten. Klanten zullen profiteren van een overzichtelijke klantenzone, terwijl medewerkers efficiënter kunnen samenwerken dankzij de resource library en real-time bewerkingsmogelijkheden.
\\
\\
Dit project zal agencies in staat stellen om hun processen te stroomlijnen, de communicatie te verbeteren, en de klanttevredenheid te verhogen. Het platform biedt niet alleen een oplossing voor bestaande problemen, maar ook een toekomstgerichte aanpak voor web agencies die willen groeien en innoveren.


%%---------- Andere bijlagen --------------------------------------------------
% TODO: Voeg hier eventuele andere bijlagen toe. Bv. als je deze BP voor de
% tweede keer indient, een overzicht van de verbeteringen t.o.v. het origineel.
%\input{...}

%%---------- Backmatter, referentielijst ---------------------------------------

\backmatter{}

\setlength\bibitemsep{2pt} %% Add Some space between the bibliograpy entries
\printbibliography[heading=bibintoc]

\end{document}
