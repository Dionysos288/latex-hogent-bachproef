%%=============================================================================
%% Methodologie
%%=============================================================================

\chapter{\IfLanguageName{dutch}{Methodologie}{Methodology}}
\label{ch:methodologie}

Dit hoofdstuk beschrijft de methodologische aanpak die is gebruikt voor het onderzoek en de ontwikkeling van het samenhangend platform. Het onderzoek is uitgevoerd in verschillende fasen, elk met specifieke doelstellingen en methoden.

\section{Onderzoeksfasen}
\label{sec:onderzoeksfasen}

Het onderzoek is opgedeeld in vijf hoofdfasen:

\begin{enumerate}
    \item Literatuuronderzoek en requirements analyse
    \item Design en architectuur planning
    \item Ontwikkeling en implementatie
    \item Testing en evaluatie
    \item Analyse en conclusievorming
\end{enumerate}

\section{Fase 1: Literatuuronderzoek en Requirements}
\label{sec:fase1}

De eerste fase bestond uit een uitgebreid literatuuronderzoek naar bestaande oplossingen en best practices. Hierbij zijn verschillende onderzoeksmethoden toegepast:

\begin{itemize}
    \item Analyse van academische literatuur
    \item Analyse van bestaande platforms en hun beperkingen
    \item Interviews met potentiële eindgebruikers voor requirements gathering
\end{itemize}

Deze fase resulteerde in een gedetailleerd benodigdheden document en een duidelijk beeld van de huidige stand van zaken in het vakgebied.

\section{Fase 2: Design en Architectuur}
\label{sec:fase2}

De tweede fase focuste op het ontwerp van de technische architectuur en gebruikersinterface:

\begin{itemize}
    \item Ontwikkeling van technische architectuur gebaseerd op Next.js
    \item Design van database schema's met Prisma
    \item Opstellen van UI/UX wireframes en prototypes
    \item Planning van integraties met externe systemen
\end{itemize}

Hierbij is gebruik gemaakt van moderne design methodologieën zoals een feedback sessie met de co promotor.

\section{Fase 3: Ontwikkeling}
\label{sec:fase3}

De ontwikkelingsfase volgde een aanpak met een zoveel mogelijk doen op het moment dat de code geschreven wordt:

\begin{itemize}
    \item Implementatie van core functionaliteiten:
    \begin{itemize}
        \item Project en task management systeem
        \item Resource library met code editor
        \item Klantportaal met fase-tracking
        \item Calendar integratie
    \end{itemize}
\end{itemize}

\section{Fase 4: Testing en Evaluatie}
\label{sec:fase4}

De evaluatiefase bestond uit verschillende testmethodologieën:

\begin{itemize}
    \item Functionele tests van alle componenten
    \item Usability testing met eindgebruikers
    \item Security audits van het platform
\end{itemize}

Feedback werd verzameld via:
\begin{itemize}
    \item gebruikersinterviews
    \item Performance metrics analyse
    \item feedback sessie met de co promotor
\end{itemize}

\section{Fase 5: Analyse en Conclusie}
\label{sec:fase5}

De laatste fase focuste op het analyseren van de verzamelde data en het trekken van conclusies:

\begin{itemize}
    \item Analyse van gebruikersfeedback en testresultaten
    \item Vergelijking met initiële requirements
    \item Identificatie van verbeterpunten en aanbevelingen
    \item Documentatie van bevindingen en conclusies
\end{itemize}

Deze methodologische aanpak heeft geleid tot een systematische ontwikkeling en evaluatie van het platform, waarbij elke fase heeft bijgedragen aan het beantwoorden van de onderzoeksvragen. De gedetailleerde resultaten van elke fase worden in de volgende hoofdstukken besproken.


