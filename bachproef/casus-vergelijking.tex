\chapter{Casus: Vergelijkende Analyse met Bestaande Applicaties}
\label{ch:casus-vergelijking}

In dit hoofdstuk wordt een concrete casus uitgewerkt waarin het ontwikkelde platform wordt vergeleken met enkele toonaangevende bestaande applicaties die veel gebruikt worden binnen web agencies, zoals Jira, Trello, Figma, Notion en ClickUp. Het doel is om de praktische meerwaarde, de efficiëntie en de gebruikerservaring van het eigen platform te toetsen aan de hand van een realistisch projectscenario.

\section{Beschrijving van de casus}
\label{sec:casus-beschrijving}

Stel: een web agency krijgt de opdracht om voor een klant een nieuwe bedrijfswebsite te ontwerpen en te ontwikkelen. Het project omvat verschillende fases: intake en analyse, design, development, testing en oplevering. Het team bestaat uit een projectmanager, twee developers, een designer en een klant die de voortgang opvolgt en feedback geeft.

\section{Werken met traditionele tools}
\label{sec:casus-traditioneel}

In een klassieke workflow worden verschillende tools ingezet:
\begin{itemize}
    \item \textbf{Jira} voor het plannen en opvolgen van taken en sprints.
    \item \textbf{Figma} voor het ontwerpen van wireframes en UI-componenten.
    \item \textbf{GitHub} voor het beheren van code en versiecontrole.
    \item \textbf{Notion} voor het documenteren van projectafspraken, requirements en notulen.
    \item \textbf{E-mail/Slack} voor communicatie met de klant en het delen van updates.
\end{itemize}

Deze aanpak leidt tot de volgende uitdagingen:
\begin{itemize}
    \item Teamleden moeten voortdurend schakelen tussen verschillende applicaties, wat tijd kost en het risico op fouten vergroot.
    \item Informatie raakt verspreid: taken in Jira, designs in Figma, documentatie in Notion, code in GitHub, communicatie in Slack of e-mail.
    \item Klanten hebben geen centraal overzicht van de voortgang en moeten voor feedback verschillende kanalen gebruiken.
    \item Het zoeken naar de juiste informatie of bestanden kost veel tijd, vooral bij grotere projecten.
    \item Het bijhouden van de status van taken, assets en feedback is omslachtig en niet altijd up-to-date.
\end{itemize}

\section{Werken met het ontwikkelde platform}
\label{sec:casus-eigen-platform}

In het eigen platform worden alle bovenstaande aspecten samengebracht in één geïntegreerde omgeving:
\begin{itemize}
    \item \textbf{Project- en taakbeheer:} Alle taken, fases en deadlines zijn zichtbaar in een centraal dashboard met drag-and-drop, inline editing en bulkacties.
    \item \textbf{Resource library:} Code snippets, design assets (inclusief Figma-integratie en Cloudflare Images) en documenten zijn centraal opgeslagen, doorzoekbaar en versieerbaar.
    \item \textbf{Klantenzone:} De klant heeft een eigen portaal waar hij de voortgang kan volgen, feedback kan geven en documenten kan inzien.
    \item \textbf{Realtime notificaties:} Alle teamleden en de klant ontvangen direct updates bij relevante wijzigingen.
    \item \textbf{Integraties:} Google Calendar, Figma en Cloudflare Images zijn naadloos geïntegreerd, waardoor planning en resourcebeheer automatisch verlopen.
    \item \textbf{Samenwerking:} Teamleden kunnen in real-time samenwerken aan documenten, taken en assets zonder de context van het project te verliezen.
\end{itemize}

\section{Vergelijkende analyse: voordelen en beperkingen}
\label{sec:casus-analyse}

\subsection{Voordelen van het eigen platform}
\begin{itemize}
    \item \textbf{Efficiëntie:} Minder tijdverlies door context-switching; alle informatie is centraal beschikbaar.
    \item \textbf{Overzicht:} Eén dashboard voor alle projectinformatie, taken, assets en communicatie.
    \item \textbf{Transparantie:} Klanten zijn continu op de hoogte van de voortgang en kunnen eenvoudig feedback geven.
    \item \textbf{Samenwerking:} Real-time updates en centrale opslag bevorderen samenwerking en kennisdeling.
    \item \textbf{Gebruiksgemak:} Intuïtieve interface, snelle interacties dankzij local-first synchronisatie en bulkacties.
    \item \textbf{Beheer:} Eenvoudig rechtenbeheer en duidelijke scheiding tussen teams en klantenzones.
\end{itemize}

\subsection{Beperkingen en aandachtspunten}
\begin{itemize}
    \item \textbf{Integraties:} Hoewel de belangrijkste integraties aanwezig zijn, missen sommige agencies mogelijk koppelingen met andere tools zoals Slack of Jira.
    \item \textbf{Adoptie:} Teams die gewend zijn aan bestaande tools moeten wennen aan een nieuwe workflow.
    \item \textbf{Schaalbaarheid:} Het platform is nog niet grootschalig getest met honderden gebruikers of zeer grote projecten.
    \item \textbf{Mobiele ondersteuning:} Een mobiele app of geoptimaliseerde mobiele webversie kan de toegankelijkheid verder vergroten.
\end{itemize}

\section{Praktijkervaring en gebruikersfeedback}
\label{sec:casus-feedback}

Tijdens de testfase werd de casus in een niet echte omgeving uitgevoerd door een testteam. De gebruikers gaven aan dat zij sneller konden schakelen tussen taken, minder tijd kwijt waren aan het zoeken naar informatie en dat de communicatie met de klant soepeler verliep. Vooral de centrale resource library en de klantenzone werden als grote voordelen ervaren. Enkele testers gaven aan dat zij in de toekomst graag nog meer automatisering en integraties zouden zien, evenals een mobiele app.

\section{Conclusie van de casusvergelijking}
\label{sec:casus-conclusie}

De casus toont aan dat het ontwikkelde platform een duidelijke meerwaarde biedt ten opzichte van het werken met een versnipperd landschap van losse tools. Door alle kernfunctionaliteiten te bundelen in één omgeving, wordt de workflow van web agencies efficiënter, transparanter en gebruiksvriendelijker. Verdere optimalisatie en uitbreiding van integraties kunnen de impact van het platform in de praktijk nog verder vergroten. 