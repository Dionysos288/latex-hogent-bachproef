\chapter{\IfLanguageName{dutch}{Stand van zaken}{State of the art}}
\label{ch:stand-van-zaken}

Dit hoofdstuk presenteert een grondige analyse van de huidige staat van project- en resourcemanagement binnen web agencies. Het onderzoek richt zich op de identificatie van de grootste problemen, evaluatie van bestaande oplossingen en analyse van toekomstige ontwikkelingsmogelijkheden. Deze literatuurstudie vormt de theoretische basis voor de ontwikkeling van een samenhangend platform.

\section{Huidige situatie in web agencies}
\label{sec:huidige-situatie}

De hedendaagse web development sector wordt gekenmerkt door toenemende complexiteit in projectmanagement en resource allocatie. Het \textcite{GitLab2023} DevSecOps rapport identificeert verschillende kritieke uitdagingen waarmee development teams dagelijks worden geconfronteerd:

\begin{itemize}
    \item \textbf{Tool fragmentatie}: Teams moeten schakelen tussen veel verschillende systemen
    \item \textbf{Workflow onderbrekingen}: Constante context-switching tussen tools
    \item \textbf{Kennisverspreiding}: Informatie verspreid over meerdere platforms
\end{itemize}

\subsection{Project en task management}
\label{subsec:project-management}

Het \textcite{Atlassian2023} rapport over Agile practices identificeert verschillende uitdagingen in modern projectmanagement:

\begin{itemize}
    \item \textbf{Project Planning}
    \begin{itemize}
        \item Moeilijkheden bij het opstellen van realistische timelines
        \item Uitdagingen in het definiëren van project scopes
        \item Complexiteit van budget allocatie
        \item Gebrek aan inzicht in resource beschikbaarheid
    \end{itemize}
    
    \item \textbf{Task Management}
    \begin{itemize}
        \item Uitdagingen in het prioriteren van taken
        \item Problemen met het tracken van afhankelijkheden
        \item Moeilijkheden bij het balanceren van workload
        \item Gebrek aan duidelijke task ownership
    \end{itemize}
    
    \item \textbf{Team Resource Allocatie}
    \begin{itemize}
        \item Complexiteit van skill-matching met taken
        \item Uitdagingen in capacity planning
        \item Moeilijkheden bij het managen van parallelle projecten
        \item Gebrek aan flexibiliteit bij onverwachte wijzigingen
    \end{itemize}
    
    \item \textbf{Project Monitoring}
    \begin{itemize}
        \item Beperkt inzicht in real-time projectstatus
        \item Moeilijkheden bij het tracken van milestones
        \item Uitdagingen in het meten van project gezondheid
        \item Gebrek aan vroege waarschuwingssystemen
    \end{itemize}
\end{itemize}

\subsection{Resource Library Management}
\label{subsec:resource-library}

Het \textcite{GitHub2023} Octoverse rapport belicht de cruciale rol van een goed georganiseerde resource library:

\begin{itemize}
    \item \textbf{Code Management}
    \begin{itemize}
        \item Uitdagingen in het organiseren van herbruikbare componenten
        \item Moeilijkheden bij het vinden van relevante code snippets
        \item Gebrek aan context bij bestaande oplossingen
        \item Versie beheer van gedeelde componenten
        \item Documentatie van code gebruik en implementatie
    \end{itemize}
    
    \item \textbf{Design Asset Management}
    \begin{itemize}
        \item Complexiteit van design system versioning
        \item Uitdagingen in het beheren van brand assets
        \item Moeilijkheden bij het delen van UI componenten
        \item Gebrek aan consistentie in asset gebruik
        \item Inefficiënt zoeken naar specifieke assets
    \end{itemize}
    
    \item \textbf{Document Management}
    \begin{itemize}
        \item Uitdagingen in template standardisatie
        \item Moeilijkheden bij het tracken van document versies
        \item Gebrek aan gestructureerde metadata
        \item Inefficiënte zoek- en filtermogelijkheden
        \item Beperkte mogelijkheden voor collaborative editing
    \end{itemize}
\end{itemize}

\subsection{Integratie Uitdagingen}
\label{subsec:integratie}

Het \textcite{StackOverflow2023} onderzoek identificeert specifieke uitdagingen bij het integreren van verschillende systemen:

\begin{itemize}
    \item \textbf{Data Synchronisatie}
    \begin{itemize}
        \item Inconsistenties tussen verschillende platforms
        \item Uitdagingen in real-time updates
        \item Complexiteit van data mapping
        \item Moeilijkheden bij het behouden van data integriteit
    \end{itemize}
    
    \item \textbf{Workflow Integratie}
    \begin{itemize}
        \item Gebrek aan naadloze overgangen tussen tools
        \item Uitdagingen in het automatiseren van processen
        \item Moeilijkheden bij het tracken van cross-tool activiteiten
        \item Beperkte mogelijkheden voor custom workflows
    \end{itemize}
    
    \item \textbf{User Experience}
    \begin{itemize}
        \item Inconsistente interfaces tussen verschillende tools
        \item Complexe navigatie tussen systemen
        \item Verwarrende notificatie systemen
        \item Gebrek aan unified search functionaliteit
    \end{itemize}
\end{itemize}

\section{Bestaande Oplossingen}
\label{sec:bestaande-oplossingen}

Het \textcite{GitLab2023} rapport analyseert de huidige markt van project en resource management tools:

\begin{itemize}
    \item \textbf{Project Management Platforms}
    \begin{itemize}
        \item Jira: Krachtig maar complex voor kleine teams
        \item Trello: Gebruiksvriendelijk maar beperkt in functionaliteit
        \item Asana: Goed voor task management, zwak in resource beheer
        \item Monday: Visueel sterk maar beperkt in technische integraties
        \item ClickUp: Veelzijdig maar overweldigend in opties
    \end{itemize}
    
    \item \textbf{Resource Management Tools}
    \begin{itemize}
        \item Figma: Sterk in design maar beperkt in andere resources
        \item GitHub: Uitstekend voor code, zwak in andere assets
        \item Notion: Goed voor documenten, beperkt in technische assets
        \item Confluence: Krachtig voor documentatie, complex in gebruik
        \item SharePoint: Uitgebreid maar niet ontwikkelaar-vriendelijk
    \end{itemize}
\end{itemize}

\section{Toekomstige Ontwikkelingen}
\label{sec:toekomst}

Het \textcite{StateOfAgile2023} rapport identificeert belangrijke trends voor project en resource management:

\begin{itemize}
    \item \textbf{Project Management Innovaties}
    \begin{itemize}
        \item AI-ondersteunde planning en estimatie
        \item Predictieve analytics voor projectrisico's
        \item Geautomatiseerde resource allocatie
        \item Smart workload balancing
        \item Contextbewuste task prioritering
    \end{itemize}
    
    \item \textbf{Resource Library Evolutie}
    \begin{itemize}
        \item AI-powered resource tagging en categorisatie
        \item Automatische versie controle en conflict resolutie
        \item Smart search en aanbevelingen
        \item Real-time collaborative editing
        \item Geautomatiseerde metadata generatie
    \end{itemize}
    
    \item \textbf{Integratie Verbeteringen}
    \begin{itemize}
        \item Unified platforms voor alle projectresources
        \item Naadloze cross-tool workflows
        \item Geautomatiseerde data synchronisatie
        \item Contextbewuste navigatie tussen tools
        \item Intelligente notificatie systemen
    \end{itemize}
\end{itemize}

\section{Conclusie Literatuurstudie}
\label{sec:conclusie}

Deze literatuurstudie toont aan dat er een duidelijke nood is aan een samenhangend platform voor web agencies. De huidige niet samenhangende aanpak leidt tot significante inefficiënties in projectmanagement, resource beheer en klantcommunicatie. Een succesvol platform moet de volgende aspecten combineren:

\begin{itemize}
    \item \textbf{Geïntegreerd Project Management}
    \begin{itemize}
        \item Intuïtieve planning en tracking
        \item Flexibele resource allocatie
        \item Real-time voortgangsmonitoring
        \item Geautomatiseerde workflows
    \end{itemize}
    
    \item \textbf{Centrale Resource Library}
    \begin{itemize}
        \item Unified storage voor alle assets
        \item Krachtige zoek- en filterfuncties
        \item Gestructureerd versie beheer
        \item Intelligente categorisatie
    \end{itemize}
    
    \item \textbf{Naadloze Integratie}
    \begin{itemize}
        \item Consistente gebruikerservaring
        \item Geautomatiseerde synchronisatie
        \item Contextbewuste navigatie
        \item Unified search across resources
    \end{itemize}
\end{itemize}

Deze inzichten vormen de basis voor de ontwikkeling van het platform dat in de volgende hoofdstukken wordt beschreven. De focus ligt daarbij op het creëren van een samenhangende oplossing die de gevonden problemen adresseert en aan de moderne eisen van web development teams voldoet.

