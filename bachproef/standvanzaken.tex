\chapter{\IfLanguageName{dutch}{Stand van zaken}{State of the art}}
\label{ch:stand-van-zaken}

Dit hoofdstuk presenteert een grondige analyse van de huidige staat van project- en resourcemanagement binnen web agencies. Het onderzoek richt zich op de identificatie van de grootste problemen, evaluatie van bestaande oplossingen en analyse van toekomstige ontwikkelingsmogelijkheden. Deze literatuurstudie vormt de theoretische basis voor de ontwikkeling van een samenhangend platform.

\section{Huidige situatie in web agencies}
\label{sec:huidige-situatie}

De hedendaagse web development sector wordt gekenmerkt door toenemende moeilijkheden in projectmanagement en resource allocatie. Uit een grootschalig onderzoek van \textcite{Reid2014}, waarbij 500 web agencies werden geanalyseerd, blijkt dat de effectiviteit van ontwikkelteams heel erg wordt beperkt door uit elkaar gezette tooling en niet-geoptimaliseerde workflows.

Deze bevinding wordt verder ondersteund door \textcite{Alexander2019}, die aantoont dat 73\% van de web agencies worstelt met inefficiënties door het gebruik van niet samenhangende systemen. Het onderzoek identificeert drie primaire probleemgebieden:

\begin{itemize}
    \item \textbf{Workflow fragmentatie}: Teams besteden gemiddeld 12,3 uur per week aan het wisselen tussen verschillende tools en platforms
    \item \textbf{Communicatie overhead}: 67\% van de projectvertragingen wordt veroorzaakt door miscommunicatie tussen team en klant
    \item \textbf{Resource verspilling}: 42\% van de ontwikkeltijd wordt besteed aan het zoeken naar en hergebruiken van bestaande resources
\end{itemize}

\subsection{Project en task management}
\label{subsec:project-management}

Het effectief beheren van projecten en taken vormt een uitdaging voor moderne web agencies. Een diepgaande analyse door \textcite{Taylor2024} identificeert vijf kritieke aspecten die de productiviteit beïnvloeden:

\begin{itemize}
    \item \textbf{Werklastverdeling}: De complexiteit van het toewijzen en plannen van taken binnen teams
    \begin{itemize}
        \item 64\% van de projectmanagers heeft moeite met het balanceren van workload
        \item 78\% ervaart problemen met het tracken van team capaciteit
        \item 53\% kan moeilijk inschatten hoeveel tijd taken kosten
    \end{itemize}
    
    \item \textbf{Voortgangsbewaking}: Het systematisch monitoren van project- en taakvoortgang
    \begin{itemize}
        \item Gemiddeld 4,2 uur per week besteed aan status updates
        \item 82\% van de teams heeft geen real-time inzicht in projectstatus
        \item 71\% ervaart vertraging door late identificatie van blokkades
    \end{itemize}
    
    \item \textbf{Resource allocatie}: De optimale inzet van beschikbare teamcapaciteit
    \begin{itemize}
        \item 58\% van de teams heeft geen duidelijk overzicht van beschikbare resources
        \item 43\% ervaart regelmatig overbelasting van specifieke teamleden
        \item 67\% kan moeilijk anticiperen op toekomstige resource behoeften
    \end{itemize}
\end{itemize}

Deze bevindingen worden verder onderzocht in het onderzoek van \textcite{Rodriguez2023}, dat specifiek kijkt naar projectmanagement binnen web agencies. Het onderzoek, uitgevoerd over een periode van 18 maanden bij 150 agencies, identificeert de volgende kernproblemen:

\begin{itemize}
    \item \textbf{Deadline management}
    \begin{itemize}
        \item 76\% van de projecten overschrijdt de initiële planning
        \item Gemiddelde vertraging van 3,4 weken per project
        \item 82\% van de vertragingen is gerelateerd aan communicatieproblemen
    \end{itemize}
    
    \item \textbf{Team coördinatie}
    \begin{itemize}
        \item 45\% van de ontwikkelaars werkt regelmatig aan verkeerde versies
        \item 67\% ervaart problemen met het synchroniseren van werk
        \item 58\% mist belangrijke project updates
    \end{itemize}
\end{itemize}

\subsection{Resource management en kennisdeling}
\label{subsec:resource-management}

Het effectief beheren van digitale resources vormt een kritieke uitdaging. \textcite{Chen2024a} presenteert in zijn onderzoek naar resource management systemen de volgende bevindingen:

\begin{itemize}
    \item \textbf{Code hergebruik}
    \begin{itemize}
        \item 72\% van de code is potentieel herbruikbaar
        \item Slechts 31\% wordt daadwerkelijk hergebruikt
        \item 84\% van de ontwikkelaars heeft moeite met het vinden van bestaande code
    \end{itemize}
    
    \item \textbf{Asset management}
    \begin{itemize}
        \item Gemiddeld 15,3 uur per week besteed aan het zoeken naar assets
        \item 63\% van de design files heeft duplicate versies
        \item 47\% van de assets is niet centraal toegankelijk
    \end{itemize}
    
    \item \textbf{Documentatie}
    \begin{itemize}
        \item 82\% van de projecten mist technische documentatie
        \item 56\% van de documentatie is verouderd
        \item 73\% van de teams heeft geen gestandaardiseerd documentatieproces
    \end{itemize}
\end{itemize}

\subsection{Klantcommunicatie en Projecttransparantie}
\label{subsec:klant-communicatie}

Een effectieve klantcommunicatie strategie is essentieel voor projectsucces. \textcite{Wang2024} presenteert in zijn studie over klantrelaties in web development de volgende inzichten:

\begin{itemize}
    \item \textbf{Communicatie effectiviteit}
    \begin{itemize}
        \item 76\% van de projectvertragingen is gerelateerd aan miscommunicatie
        \item Gemiddeld 8,4 uur per week besteed aan e-mail correspondentie
        \item 62\% van de klanten ervaart gebrek aan transparantie
    \end{itemize}
    
    \item \textbf{Feedback management}
    \begin{itemize}
        \item 53\% van de feedback wordt niet tijdig verwerkt
        \item 67\% van de revisierondes duurt langer dan gepland
        \item 84\% van de teams mist een gestructureerd feedbackproces
    \end{itemize}
\end{itemize}

\section{Bestaande Oplossingen}
\label{sec:bestaande-oplossingen}

De huidige markt biedt verschillende tools die deeloplossingen bieden. \textcite{Wilson2024} presenteert een uitgebreide analyse van bestaande platforms:

\begin{itemize}
    \item \textbf{Project management systemen}
    \begin{itemize}
        \item 47\% gebruikt Jira of soortgelijke tools
        \item 38\% werkt met Trello of Asana
        \item 15\% heeft eigen systemen ontwikkeld
        \item Gemiddeld 3,4 verschillende tools per team
    \end{itemize}
    
    
    \item \textbf{Client portals}
    \begin{itemize}
        \item 62\% heeft geen dedicated klantomgeving
        \item 43\% gebruikt algemene file sharing platforms
        \item 78\% mist real-time voortgangsmonitoring
    \end{itemize}
\end{itemize}

\section{Technische Vereisten}
\label{sec:technische-vereisten}

De ontwikkeling van een geïntegreerd platform vereist specifieke technische capaciteiten. \textcite{Thompson2024} analyseert de volgende aspecten:

\begin{itemize}
    \item \textbf{Performance eisen}
    \begin{itemize}
        \item Maximale laadtijd van 2 seconden per pagina
        \item Ondersteuning voor minimaal 100 simultane gebruikers
        \item Real-time synchronisatie binnen 200ms
        \item 99.9\% uptime garantie
    \end{itemize}
    
    \item \textbf{Schaalbaarheid}
    \begin{itemize}
        \item Horizontale schaalbaarheid voor groeiende teams
        \item Efficiënt geheugengebruik bij grote datasets
        \item Optimale database performance
        \item Load balancing capaciteit
    \end{itemize}
\end{itemize}

\section{Security en Privacy}
\label{sec:security}

Security vormt een kritieke component in moderne web development platforms. \textcite{Kumar2023} identificeert essentiële beveiligingsaspecten:

\begin{itemize}
    \item \textbf{Data beveiliging}
    \begin{itemize}
        \item End-to-end encryptie voor gevoelige data
        \item Multi-factor authenticatie voor toegang
        \item Role-based access control (RBAC)
        \item Regular security audits
    \end{itemize}
    
    \item \textbf{Compliance vereisten}
    \begin{itemize}
        \item GDPR compliance voor Europese markt
        \item ISO 27001 certificering
        \item Data retention policies
    \end{itemize}
\end{itemize}

\section{Toekomstige Ontwikkelingen}
\label{sec:toekomst}

De evolutie van web development platforms wordt gedreven door verschillende technologische trends. \textcite{Chen2024a} identificeert de volgende ontwikkelingen:

\begin{itemize}
    \item \textbf{AI-integratie}
    \begin{itemize}
        \item Automatische code suggesties
        \item Predictive project planning
        \item Smart resource allocation
    \end{itemize}
    
    
    \item \textbf{Collaboratie tools}
    \begin{itemize}
        \item Integrated communication
    \end{itemize}
\end{itemize}

\section{Conclusie Literatuurstudie}
\label{sec:conclusie}

Deze literatuurstudie toont aan dat er een duidelijke nood is aan een samenhangend platform voor web agencies. De huidige niet samenhangende aanpak leidt tot significante inefficiënties in projectmanagement, resource beheer en klantcommunicatie. Een succesvol platform moet de volgende aspecten combineren:

\begin{itemize}
    \item Effectief project- en taskmanagement
    \item Geïntegreerde code editing
    \item Centrale resource bibliotheek
    \item Transparante klantcommunicatie
    \item Robuuste security maatregelen
\end{itemize}

Deze inzichten vormen de basis voor de ontwikkeling van het platform dat in de volgende hoofdstukken wordt beschreven. De focus ligt daarbij op het creëren van een samenhangende oplossing die de gevonden problemen adresseert en aan de moderne eisen van web development teams voldoet.

