%%=============================================================================
%% Samenvatting
%%=============================================================================

% TODO: De "abstract" of samenvatting is een kernachtige (~ 1 blz. voor een
% thesis) synthese van het document.
%
% Een goede abstract biedt een kernachtig antwoord op volgende vragen:
%
% 1. Waarover gaat de bachelorproef?
% 2. Waarom heb je er over geschreven?
% 3. Hoe heb je het onderzoek uitgevoerd?
% 4. Wat waren de resultaten? Wat blijkt uit je onderzoek?
% 5. Wat betekenen je resultaten? Wat is de relevantie voor het werkveld?
%
% Daarom bestaat een abstract uit volgende componenten:
%
% - inleiding + kaderen thema
% - probleemstelling
% - (centrale) onderzoeksvraag
% - onderzoeksdoelstelling
% - methodologie
% - resultaten (beperk tot de belangrijkste, relevant voor de onderzoeksvraag)
% - conclusies, aanbevelingen, beperkingen
%
% LET OP! Een samenvatting is GEEN voorwoord!

%%---------- Nederlandse samenvatting -----------------------------------------
%
% TODO: Als je je bachelorproef in het Engels schrijft, moet je eerst een
% Nederlandse samenvatting invoegen. Haal daarvoor onderstaande code uit
% commentaar.
% Wie zijn bachelorproef in het Nederlands schrijft, kan dit negeren, de inhoud
% wordt niet in het document ingevoegd.

\IfLanguageName{english}{%
\selectlanguage{dutch}
\chapter*{Samenvatting}
\lipsum[1-4]
\selectlanguage{english}
}{}

%%---------- Samenvatting -----------------------------------------------------
% De samenvatting in de hoofdtaal van het document

\chapter*{\IfLanguageName{dutch}{Samenvatting}{Abstract}}

% Web agencies gebruiken veel verschillende tools voor hun dagelijks werk. Ze hebben aparte systemen voor projectbeheer, code schrijven, bestanden delen en klantcommunicatie. Dit zorgt voor problemen: het kost veel tijd om tussen deze tools te wisselen, informatie raakt verspreid en het is moeilijk om overzicht te houden.

% Deze bachelorproef onderzoekt hoe we deze problemen kunnen oplossen door één platform te maken dat alle nodige functies combineert. Het doel is om het werk van web agencies makkelijker en efficiënter te maken.

% De hoofdvraag van dit onderzoek is: "Hoe kunnen we één systeem maken dat projectbeheer, resource management en klantcommunicatie samenbrengt voor web agencies?"

% Om deze vraag te beantwoorden hebben we eerst onderzocht welke problemen web agencies nu hebben. Daarna hebben we een platform ontwikkeld met deze belangrijke functies:
% \begin{itemize}
%     \item Een projectbeheersysteem met takenlijsten en planning
%     \item Een bibliotheek voor het delen van code en bestanden
%     \item Een klantportaal waar klanten projectvoortgang kunnen zien
%     \item Een systeem dat werkt met Google Calendar
%     \item Een code editor waar teams samen kunnen werken
% \end{itemize}

% We hebben het platform getest met verschillende web agencies. Uit de tests blijkt dat:
% \begin{itemize}
%     \item Teams 40\% minder tijd kwijt zijn aan het wisselen tussen tools
%     \item Klanten beter op de hoogte blijven van hun projecten
%     \item Het makkelijker is om bestanden en code te delen
%     \item De communicatie tussen team en klant veel beter verloopt
% \end{itemize}

% Deze resultaten laten zien dat een geïntegreerd platform echt kan helpen om het werk van web agencies te verbeteren. Het platform zorgt voor betere samenwerking, minder tijdverlies en meer tevreden klanten. Voor de toekomst raden we aan om het platform uit te breiden met AI-functies en nog meer integraties met andere tools.
